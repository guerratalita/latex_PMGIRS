% Please add the following required packages to your document preamble:
% \usepackage[table,xcdraw]{xcolor}
% If you use beamer only pass "xcolor=table" option, i.e. \documentclass[xcolor=table]{beamer}
% \usepackage{longtable}
% Note: It may be necessary to compile the document several times to get a multi-page table to line up properly
\begin{longtable}{
		>{\columncolor[HTML]{F79646}}P{0.2\textwidth} 
		>{\columncolor[HTML]{FFCE93}}P{0.3\textwidth} 
		>{\columncolor[HTML]{FFFC9E}}P{0.3\textwidth}
		>{\columncolor[HTML]{FFFFC7}}P{0.2\textwidth} } 
	\caption{}
	%\arrayrulecolor{black}
	\label{tab:my-table}\\
	\textbf{Objetivos} & \textbf{Metas} & \textbf{Ações} & \textbf{Prazos}
	\endfirsthead
	%
	\multicolumn{4}{c}%
	{{\bfseries Tabela \thetable\ : continuação da página anterior}} \\
	\textbf{Objetivos} & \textbf{Metas} & \textbf{Ações} & \textbf{Prazos} \\
	\endhead
	%
	Reduzir a geração de RSU & Aumentar em 60 \% a qualidade da segregação dos RS comum & Divulgação constante & 2020-2040 \\
	&  & Afixar nos prédios públicos as categorias de resíduos e formas de descarte & 2022\\
	& Reduzir em 30 \% o que for destinado a aterro (como RSU) & Incentivo à compostagem doméstica e coletiva & 2020-2040 \\
	&  & Criar pontos de recebimento de roupas e tecidos & 2022 \\
	&  & Melhorar a segregação na fonte de geração & 2020-2040 \\
	& Coleta seletiva & Recipientes para segregação correta em todos os prédios públicos & 2024 \\
	&  & Incentivo a programa de pontos & 2028 \\
	&  & Divulgação constante & 2020-2040 \\
	&  & Expandir os pontos de coleta de óleo de cozinha usado (centros comunitários, escolas e mercados, em cada bairro) & 2022 \\
	& Compostagem & Compostar, localmente, 35 \% dos orgânicos gerados no município & 2028 \\
	&  & Composteiras em todos os prédios públicos & 2022 \\
	&  & Ponto de armazenagem de matéria seca para compostagem & 2022 \\
	Promover gestão participativa da comunidade (destaca-se aqui a importância de valorizar o conhecimento da população local) & Criar e manter canais de comunicação abertos com ênfase na Educação Ambiental & Manter as redes sociais ativas & 2020-2040 \\
	&  & Promover, periodicamente, rodas de conversa com munícipes para melhoria do diálogo com a sociedade & 2020-2040 \\
	&  & Incentivar o engajamento contínuo da rede escolar presente no município & 2020-2040 \\
	&  & Priorização da educação ambiental nos currículos escolares & 2020-2040 \\
	&  & Promover, periodicamente, oficinas participativas com a sociedade & 2020-2040 \\
	Fomentar ações que possibilitem geração de renda via resíduos & Organizar uma associação ou cooperativa de catadores de materiais reutilizáveis ou recicláveis, respeitando as capacidades de liderança dos próprios cooperados & Dialogar com os catadores autônomos & 2022 \\
	&  & Consulta ao Programa Pró-Catador para obtenção de recursos para a infraestrutura & 2022 \\
	&  & Estudo sobre possíveis terrenos públicos onde a cooperativa/associação poderá ser instalada & 2022 \\
	&  & Incentivar a criação por meio de abatimento de impostos ou linhas de crédito diferenciadas & 2024 \\
	&  & Construção da cooperativa/associação & 2024 \\
	& Incentivos a empresas que utilizem, no mínimo, o princípio dos 3R & Abatimento de impostos, linhas de crédito diferenciadas & 2020-2040 \\
	& Incentivar a criação/instalação de empresas que façam o beneficiamento (agregar valor) aos resíduos provenientes da coleta seletiva & Abatimento de impostos, linhas de crédito diferenciadas & 2020-2040 \\
	Melhoria dos serviços de limpeza urbana e manejo de RS & Padronização e instalação de "lixeiras" & Definição de modelo mais ergonômico aos coletores (com porta articulada, por exemplo), duradouro e com separação para orgânico e reciclável. Preferencialmente com cobertura contra chuvas. & 2020 \\
	& Otimizar as rotas dos caminhões coletores & Estudo para identificar possíveis rotas mais eficientes e manter o padrão desse deslocamento para geraçao de dados consistentes a respeito & 2024 \\
	& Manutenção preventiva dos equipamentos & Criar uma escala de manutenção preventiva & 2020 \\
	& Registro de dados de movimentação dos resíduos & Manter o controle quantitativo da movimentação dos resíduos & 2020 \\
	Correção do gerenciamento dos RCC & Regularizar terreno para disposição de RCC & Interromper a disposição de RCC para o "bota-fora" municipal & 2022 \\
	&  & Contatar o órgão ambiental competente para verificar o procedimento & 2020 \\
	& Criar meios de reaproveitamento de RCC & Criação de pontos de entrega, armazenagem e troca de RCC & 2022 \\
	&  & Estimular intercâmbio de materiais RCC entre os munícipes & 2020-2040 \\
	Logística Reversa & Incentivar a implantação da logística reversa pós-consumo & Verificar a possibilidade do estabelecimento de parcerias e/ou programa de pontos com empreendimentos & 2022-2024 \\
	&  & Inserir pontos de coleta em lugares de fácil acesso & 2022-2024 \\
	Preencher correta e regularmente os formulários do SNIS & Criar um padrão de coleta dos dados para evitar futuras interpretações ambíguas & Verificar os dados solicitados & 2020-2040 \\
	&  & Manter o registro das informações em meio digital de modo acessivel ao responsável na gestão corrente & 2020-2040 \\
	&  & Evitar estimar dados, utilizar valores concretos (exemplo: valores de contratos, pesos e medidas coletados, etc.) & 2020-2040 \\
	&  & Utilizar valores atuais (correntes no ano) & 2020-2040 \\
	& Aprimorar anualmente o preenchimeto & Coletar durante o ano os dados solicitados & 2020-2040
\end{longtable}