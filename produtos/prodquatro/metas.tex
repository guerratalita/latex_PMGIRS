% Table generated by Excel2LaTeX from sheet 'Planilha1'
%\begin{table}[htbp]
\begin{center}
\begin{longtable}{|p{0.2\textwidth}|p{0.3\textwidth}|p{0.3\textwidth}|p{0.2\textwidth}|}
%	\centering
	\arrayrulecolor{black}
%	\caption{Add caption}
%	\begin{tabular}{|P{6.145em} | P{16.575em} | P{12.355em} | P{5.645em}|}

\hline
Objetivos & Metas & Ações & Prazos \\
\hline
\endfirsthead
%
\multicolumn{4}{c}%
{{\bfseries Table \thetable\ continued from previous page}} \\
\hline
Objetivos & Metas & Ações & Prazos \\
\hline
\endhead
%

%		\toprule
%		Objetivos & Metas & Ações & Prazos \\
%		\midrule
		\multirow{12}[24]{*}{Reduzir a geração de RSU} & \multirow{2}[4]{*}{Aumentar em 60 \% a qualidade da segregação dos RS comum} & Divulgação constante & 2020-2040 \\
		\cmidrule{3-4}          &       & Afixar nos prédios públicos as categorias de resíduos e formas de descarte & 2022 \\
		\cmidrule{2-4}          & \multirow{3}[6]{*}{Reduzir em 30 \% o que for destinado a aterro (como RSU)} & Incentivo à compostagem doméstica e coletiva & 2020-2040 \\
		\cmidrule{3-4}          &       & Criar pontos de recebimento de roupas e tecidos & 2022 \\
		\cmidrule{3-4}          &       & Melhorar a segregação na fonte de geração & 2020-2040 \\
		\cmidrule{2-4}          & \multirow{4}[8]{*}{Coleta seletiva} & Recipientes para segregação correta em todos os prédios públicos & 2024 \\
		\cmidrule{3-4}          &       & Incentivo a programa de pontos & 2028 \\
		\cmidrule{3-4}          &       & Divulgação constante & 2020-2040 \\
		\cmidrule{3-4}          &       & Expandir os pontos de coleta de óleo de cozinha usado (centros comunitários, escolas e mercados, em cada bairro) & 2022 \\
		\cmidrule{2-4}          & \multirow{3}[6]{*}{Compostagem} & Compostar, localmente, 35 \% dos orgânicos gerados no município & 2028 \\
		\cmidrule{3-4}          &       & Composteiras em todos os prédios públicos & 2022 \\
		\cmidrule{3-4}          &       & Ponto de armazenagem de matéria seca para compostagem & 2022 \\
		\midrule
		\multirow{5}[10]{*}{Promover gestão participativa da comunidade (destaca-se aqui a importância de valorizar o conhecimento da população local)} & \multirow{5}[10]{*}{Criar e manter canais de comunicação abertos com ênfase na Educação Ambiental} & Manter as redes sociais ativas & 2020-2040 \\
		\cmidrule{3-4}          &       & Promover, periodicamente, rodas de conversa com munícipes para melhoria do diálogo com a sociedade & 2020-2040 \\
		\cmidrule{3-4}          &       & Incentivar o engajamento contínuo da rede escolar presente no município & 2020-2040 \\
		\cmidrule{3-4}          &       & Priorização da educação ambiental nos currículos escolares & 2020-2040 \\
		\cmidrule{3-4}          &       & Promover, periodicamente, oficinas participativas com a sociedade & 2020-2040 \\
		\midrule
		\multirow{7}[14]{*}{Fomentar ações que possibilitem geração de renda via resíduos} & \multirow{5}[10]{*}{Organizar uma associação ou cooperativa de catadores de materiais reutilizáveis ou recicláveis, respeitando as capacidades de liderança dos próprios cooperados} & Dialogar com os catadores autônomos & 2022 \\
		\cmidrule{3-4}          &       & Consulta ao Programa Pró-Catador para obtenção de recursos para a infraestrutura & 2022 \\
		\cmidrule{3-4}          &       & Estudo sobre possíveis terrenos públicos onde a cooperativa/associação poderá ser instalada & 2022 \\
		\cmidrule{3-4}          &       & Incentivar a criação por meio de abatimento de impostos ou linhas de crédito diferenciadas & 2024 \\
		\cmidrule{3-4}          &       & Construção da cooperativa/associação & 2024 \\
		\cmidrule{2-4}          & Incentivos a empresas que utilizem, no mínimo, o princípio dos 3R & Abatimento de impostos, linhas de crédito diferenciadas & 2020-2040 \\
		\cmidrule{2-4}          & Incentivar a criação/instalação de empresas que façam o beneficiamento (agregar valor) aos resíduos provenientes da coleta seletiva & Abatimento de impostos, linhas de crédito diferenciadas & 2020-2040 \\
		\midrule
		\multirow{4}[8]{*}{Melhoria dos serviços de limpeza urbana e manejo de RS} & Padronização e instalação de "lixeiras" & Definição de modelo mais ergonômico aos coletores (com porta articulada, por exemplo), duradouro e com separação para orgânico e reciclável. Preferencialmente com cobertura contra chuvas. & 2020 \\
		\cmidrule{2-4}          & Otimizar as rotas dos caminhões coletores & Estudo para identificar possíveis rotas mais eficientes e manter o padrão desse deslocamento para geraçao de dados consistentes a respeito & 2024 \\
		\cmidrule{2-4}          & Manutenção preventiva dos equipamentos & Criar uma escala de manutenção preventiva & 2020 \\
		\cmidrule{2-4}          & Registro de dados de movimentação dos resíduos & Manter o controle quantitativo da movimentação dos resíduos & 2020 \\
		\midrule
		\multirow{4}[8]{*}{Correção do gerenciamento dos RCC} & \multirow{2}[4]{*}{Regularizar terreno para disposição de RCC} & Interromper a disposição de RCC para o "bota-fora" municipal & 2022 \\
		\cmidrule{3-4}          &       & Contatar o órgão ambiental competente para verificar o procedimento & 2020 \\
		\cmidrule{2-4}          & \multirow{2}[4]{*}{Criar meios de reaproveitamento de RCC} & Criação de pontos de entrega, armazenagem e troca de RCC & 2022 \\
		\cmidrule{3-4}          &       & Estimular intercâmbio de materiais RCC entre os munícipes & 2020-2040 \\
		\midrule
		\multirow{2}[4]{*}{Logística Reversa} & \multirow{2}[4]{*}{Incentivar a implantação da logística reversa pós-consumo} & Verificar a possibilidade do estabelecimento de parcerias e/ou programa de pontos com empreendimentos & 2022-2024 \\
		\cmidrule{3-4}          &       & Inserir pontos de coleta em lugares de fácil acesso & 2022-2024 \\
		\midrule
		\multirow{6}[12]{*}{Preencher correta e regularmente os formulários do SNIS} & \multirow{4}[8]{*}{Criar um padrão de coleta dos dados para evitar futuras interpretações ambíguas} & Verificar os dados solicitados & 2020-2040 \\
		\cmidrule{3-4}          &       & Manter o registro das informações em meio digital de modo acessivel ao responsável na gestão corrente & 2020-2040 \\
		\cmidrule{3-4}          &       & Evitar estimar dados, utilizar valores concretos (exemplo: valores de contratos, pesos e medidas coletados, etc.) & 2020-2040 \\
		\cmidrule{3-4}          &       & Utilizar valores atuais (correntes no ano) & 2020-2040 \\
		\cmidrule{2-4}          & \multirow{2}[4]{*}{Aprimorar anualmente o preenchimeto} & Coletar durante o ano os dados solicitados & 2020-2040 \\
		\cmidrule{3-4}          &       & Verificar as dificuldades enfrentadas para o preenchimento (ausência de dados, dificuldade de coleta, etc) e montar uma estratégia para solucioná-las no ano seguinte & 2020-2040 \\
		\bottomrule
%	\end{tabular}%
	\label{tab:addlabel}%
\end{longtable}%
\end{center}