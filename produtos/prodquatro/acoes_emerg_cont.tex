% Table generated by Excel2LaTeX from sheet 'Planilha1'
\begin{table}[htbp]
  \centering
  \arrayrulecolor{white}
  \caption{Ações de emergência e contigência para  município de Monteiro Lobato}
  \resizebox{\textwidth}{!}{%}
    \begin{tabular}{P{12.93em}|P{10.645em}|P{18em}}
    \rowcolor[rgb]{ .941,  .635,  .18} \multicolumn{1}{c|}{\textcolor[rgb]{ 1,  1,  1}{\textbf{Ocorrência}}} & \multicolumn{1}{c|}{\textcolor[rgb]{ 1,  1,  1}{\textbf{Origem}}} & \multicolumn{1}{c}{\textcolor[rgb]{ 1,  1,  1}{\textbf{Ações de emergência e contigência}}} \\
    \rowcolor[rgb]{ .976,  .855,  .675} Inoperância do caminhão de resíduos & Falha na parte mecânica & \begin{itemize} \item Providenciar imediatamente o reparo do equipamento \item Contatar outro município para ver a disponibilidade do caminhão de recolher os resíduos de Monteiro Lobato (rotas em comum)\item Realizar manutenção preventiva no veículo coletor 
    \end{itemize}	\\
    \rowcolor[rgb]{ .988,  .929,  .839} Paralisação dos serviços de coleta convencional & Greve dos funcionários ou da empresa responsável pelo serviços ou dos funcionários/servidores da prefeitura & \begin{itemize} \item Informar oficialmente a população para que colabora\item Negociação com funcionários paralisados\item Contatar empresa especializada (caráter emergencial)\item Realizar um cadastro de pessoas interessadas caso essa situação ocorra \end{itemize} \\
    \rowcolor[rgb]{ .976,  .855,  .675} Inoperância do Lugar de Entrega Voluntária (LEV) & Mau uso dos LEV’S por parte da população - vandalismo ou disposição erradas dos resíduos sólidos & \begin{itemize} \item Realizar manutenção preventiva no local\item Providenciar imediatamente o reparo da estrutura\item Comunicar a polícia\item Reforçar a importância dos LEV’s e seus impactos caso haja falha no processo (educação ambiental com a sociedade) \end{itemize} \\
    \rowcolor[rgb]{ .988,  .929,  .839} Inoperância do Ponto de Entrega Voluntária (PEV) & Mau uso dos PEV’S por parte da população - vandalismo ou disposição erradas dos resíduos sólidos & \begin{itemize} \item Realizar manutenção preventiva no local\item Providenciar imediatamente o reparo da estrutura\item Comunicar a polícia\item Reforçar a importância dos LEV’s e seus impactos caso haja falha no processo (educação ambiental com a sociedade)\item Identificar as empresas que podem retirar o material em caso de emergência\item Sinalizar com placas os tipos de materiais aceitos \end{itemize} \\
    \rowcolor[rgb]{ .976,  .855,  .675} Paralisação do aterro sanitário do município de Tremembé & \begin{itemize} \item Ruptura de taludes, vazamento\item Obstrução das vias de chegada ou saída\item Greve geral dos funcionários\item Quebra de contrato \item Esgotamento da área de disposição \end{itemize} & \begin{itemize} \item Realizar um estudo de possíveis locais que possam armazenar os resíduos de forma provisória\item Informar oficialmente à população, para que colabore até a situação normalizar\item Contatar aterros privados próximos a fim de firmar um contrato caso ocorra eventos emergenciais\item Negociação com funcionários paralisados \end{itemize} \\
    \end{tabular}% 
}
  \label{tab:acoes_emerg_cont}%
  \legend{Fonte: Elaborado pelos autores, adaptado do PMGIRS/Arujá (2018)}
\end{table}%
