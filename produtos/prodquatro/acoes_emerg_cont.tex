% Table generated by Excel2LaTeX from sheet 'acoes_emerg_cont'
\begin{table}[htbp]
	\centering
	\caption{Ações de emergência e contigência para  município de Monteiro Lobato}
 	 \resizebox{\textwidth}{!}{%
	
	\begin{tabular}{p{11.275em}p{11.275em}p{25.545em}}
		\rowcolor[rgb]{ .969,  .588,  .275} \textcolor[rgb]{ 1,  1,  1}{\textbf{Ocorrência}} & \textcolor[rgb]{ 1,  1,  1}{\textbf{Origem}} & \textcolor[rgb]{ 1,  1,  1}{\textbf{Ações de emergência e contingência}} \\
		\rowcolor[rgb]{ .992,  .914,  .851} Inoperância do caminhão de resíduos & Falha na parte mecânica & · Providenciar imediatamente o reparo do equipamento\newline{}· Contatar outro município para ver a disponibilidade do caminhão de recolher os resíduos de Monteiro Lobato (rotas em comum)\newline{}· Realizar manutenção preventiva no veículo coletor \\
		\rowcolor[rgb]{ .984,  .831,  .706} Paralisação dos serviços de coleta convencional & Greve dos funcionários ou da empresa responsável pelo serviços ou dos funcionários/servidores da prefeitura & · Informar oficialmente a população para que colabora\newline{}· Negociação com funcionários paralisados\newline{}· Contatar empresa especializada (caráter emergencial)\newline{}· Realizar um cadastro de pessoas interessadas caso essa situação ocorra \\
		\rowcolor[rgb]{ .992,  .914,  .851} Inoperância do Lugar de Entrega Voluntária (LEV) & Mau uso dos LEV’S por parte da população - vandalismo ou disposição erradas dos resíduos sólidos & · Realizar manutenção preventiva no local\newline{}· Providenciar imediatamente o reparo da estrutura\newline{}· Comunicar a polícia\newline{}· Reforçar a importância dos LEV’s e seus impactos caso haja falha no processo (educação ambiental com a sociedade) \\
		\rowcolor[rgb]{ .984,  .831,  .706} Inoperância do Ponto de Entrega Voluntária (PEV) & Mau uso dos PEV’S por parte da população - vandalismo ou disposição erradas dos resíduos sólidos & · Realizar manutenção preventiva no local\newline{}· Providenciar imediatamente o reparo da estrutura\newline{}· Comunicar a polícia\newline{}· Reforçar a importância dos LEV’s e seus impactos caso haja falha no processo (educação ambiental com a sociedade)\newline{}· Identificar as empresas que podem retirar o material em caso de emergência\newline{}· Sinalizar com placas os tipos de materiais aceitos \\
		\rowcolor[rgb]{ .992,  .914,  .851} Paralisação do aterro sanitário do município de Tremembé & Ruptura de taludes, vazamento\newline{}Obstrução das vias de chegada ou saída\newline{}Greve geral dos funcionários\newline{}Quebra de contrato\newline{}esgotamento da área de disposição & · Realizar um estudo de possíveis locais que possam armazenar os resíduos de forma provisória\newline{}· Informar oficialmente à população, para que colabore até a situação normalizar\newline{}· Contatar aterros privados próximos a fim de firmar um contrato caso ocorra eventos emergenciais\newline{}· Negociação com funcionários paralisados \\
	\end{tabular}%
}
  \label{tab:acoes_emerg_cont}%
  \legend{Fonte: Elaborado pelos autores, adaptado do PMGIRS/Arujá (2018)}
\end{table}%
