% Table generated by Excel2LaTeX from sheet 'cooper_associa'
\begin{table}[htbp]
  \centering
  \arrayrulecolor{white}
  \caption{Diferenças entre cooperativas e associações.}
  %\resizebox{\textwidth}{!}{
    \begin{tabular}{P{17.955em}|P{17.955em}}
    \rowcolor[rgb]{ .969,  .588,  .275} Cooperativa & Associação \\
    \rowcolor[rgb]{ .992,  .914,  .851} Os participantes são os donos do patrimônio e os beneficiários dos ganhos & Os associados não são propriamente os donos \\
    \rowcolor[rgb]{ .984,  .831,  .706} Beneficia os próprios cooperados & O patrimônio acumulado, no caso de sua dissolução, deve ser destinado a outra instituição semelhante, conforme determina a lei \\
    \rowcolor[rgb]{ .992,  .914,  .851} Por meio de assembleia geral, as sobras das relações comerciais, podem ser distribuídas entre os cooperados & Os ganhos devem ser destinados à sociedade, e não aos associados \\
    \rowcolor[rgb]{ .984,  .831,  .706} Existe o repasse dos valores relacionados ao trabalho prestado pelos cooperados ou da venda dos produtos entregues na cooperativa & Na maioria das vezes, os associados não são nem mesmo os beneficiários da ação do trabalho da associação \\
    \rowcolor[rgb]{ .992,  .914,  .851} Mínimo de 20 pessoas & Mínimo de 2 pessoas \\
    \rowcolor[rgb]{ .984,  .831,  .706} Tem capital social (formado por quotas, podendo receber doações, empréstimos e processos de capitalização), o que facilita financiamentos em instituições financeiras  & Patrimônio formado por taxas pagas pelos associados, doações,  fundos e reservas. Não possui capital social \\
    \rowcolor[rgb]{ .992,  .914,  .851} Lei n° 5.764/1971; Constituição - art. 5°, de XVII a XXI, e art. 174, §2° e Código civil (Lei n° 10.406/2002)  & Constituição - art. 5°, de XVII a XXI, e art. 174, §2° e Código civil (Lei n° 10.406/2002)  \\
    \end{tabular}%
%}
  \label{tab:cooper_associa}%
  \legend{Fonte: \cite{cooper_associa}.}
\end{table}%

