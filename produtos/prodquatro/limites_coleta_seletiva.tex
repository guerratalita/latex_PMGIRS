% Table generated by Excel2LaTeX from sheet 'limites_coleta_seletiva'
\begin{table}[htbp]
  \centering
  \caption{Descrição das formas e limites da participação do poder público local na coleta seletiva}
    \begin{tabular}{P{40.855em}}
    \rowcolor[rgb]{ .992,  .914,  .851} Implantar e operar um LEV para entrega voluntária de recicláveis \\
    \rowcolor[rgb]{ .984,  .831,  .706} Estabelecer a forma correta de segregação dos RD e RC \\
    \rowcolor[rgb]{ .992,  .914,  .851} Determinar os procedimentos para o acondicionamento e descarte adequado dos materiais recicláveis \\
    \rowcolor[rgb]{ .984,  .831,  .706} Incentivar a população a realizar às boas práticas em relação a coleta seletiva \\
    \rowcolor[rgb]{ .992,  .914,  .851} Criar e priorizar a participação de cooperativas ou outro tipo de associação de catadores constituídas por pessoas físicas de baixa renda \\
    \rowcolor[rgb]{ .984,  .831,  .706} Capacitar os servidores públicos e atores sociais envolvidos na coleta seletiva \\
    \rowcolor[rgb]{ .992,  .914,  .851} Disponibilizar lixeiras específicas para cada tipo de material reciclável em pontos estratégicos no município \\
    \rowcolor[rgb]{ .984,  .831,  .706} Formentar a implementação de soluções consorciadas ou compartilhadas com municípios vizinhos \\
    \end{tabular}%
  \label{tab:limites_coleta_seletiva}%
  \legend{Fonte: Elaborado pelos autores, adaptado da PNRS (2010).}
\end{table}%
