% Table generated by Excel2LaTeX from sheet 'fatores_aterro'
\begin{table}[htbp]
  \centering
  \caption{Fatores de análise para avaliação da área a ser implantado um aterro.}
    \begin{tabular}{P{40.955em}}
    \rowcolor[rgb]{ .992,  .914,  .851} \textbf{Tipo, consistência e granulometria das camadas de subsolo na base do aterro;} recomenda-se a utilização de solos naturalmente pouco permeáveis (solos argilosos, argilo-arenosos ou argilo-siltosos) \\
    \rowcolor[rgb]{ .984,  .831,  .706}\textbf{ No caso de existência de corpos d'água superficiais na área ou em entorno imdiato;} recomenda-se o respeito a uma distância mínima de 200 m de qualquer coleção hídrica ou curso d'água \\
    \rowcolor[rgb]{ .992,  .914,  .851} \textbf{Proximidade do freático em relação à base do aterro ou em seu entorno imediato} \\
    \rowcolor[rgb]{ .984,  .831,  .706} \textbf{Ocorrência de inundações:} as áreas com essas características não devem ser utilizadas \\
    \rowcolor[rgb]{ .992,  .914,  .851}\textbf{ Características topográficas da área devem ser tais que permitam uma das soluções adotáveis para o preenchimento do aterro,} recomenda-se locais com declividade superior a 1 \% e inferior a 30 \% \\
    \rowcolor[rgb]{ .984,  .831,  .706}\textbf{ Recomenda-se distância do limite da área útil do aterro a núcleos populacionais vizinhos mínima de 500 m} \\
    \rowcolor[rgb]{ .992,  .914,  .851} \textbf{Vida útil previsível do aterro sanitário de pequeno porte passível de ser implantado na área deve ser superior a 15 anos} \\
    \end{tabular}%
  \label{tab:fatores_aterro}%
  \legend{Fonte: \cite{abnt:13896}.}
\end{table}%
