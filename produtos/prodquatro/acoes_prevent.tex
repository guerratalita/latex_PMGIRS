% Table generated by Excel2LaTeX from sheet 'acoes_prevent'
\begin{table}[htbp]
	\centering
	\caption{Ações preventivas e corretivas do município de Monteiro Lobato.}
	\resizebox{\textwidth}{!}{%
    \begin{tabular}{p{18em}p{7.91em}p{21.545em}p{21.545em}}
	\rowcolor[rgb]{ .969,  .588,  .275} \textcolor[rgb]{ 1,  1,  1}{\textbf{AÇÃO}} & \textcolor[rgb]{ 1,  1,  1}{\textbf{PRVENTIVA (P) CORRETIVA (C)}} & \textcolor[rgb]{ 1,  1,  1}{\textbf{DIAGNÓSTICO}} & \textcolor[rgb]{ 1,  1,  1}{\textbf{PROGNÓSTICO}} \\
	\rowcolor[rgb]{ .992,  .914,  .851} Controle de emissão de gases e percolados & P     & Existente. Atualmente a disposição final dos resíduos sólidos ocorre no aterro sanitário de Tremembé, dotado de sistema de drenagem e correto tratamento/destinação dos gases e percolados, de maneira a prevenir possíveis impactos advindos dos mesmos. & Fazer frequentemente o monitoramente do aterro sanitário em que são enviados os resíduos sólidos, quanto ao licenciamento exigido pelo seu respectivo Órgão Ambiental. Analisar previamente a existência de um protocolo legal caso os resíduos sólidos passem a ser dispostos em um novo local. \\
	\rowcolor[rgb]{ .984,  .831,  .706} Educação ambiental para redução e reaproveitamento de resíduos nas fontes geradoras & P     & Existente. Nas redes de ensino. O Instituto Pandavas, localizado no Bairro do Souza, é um dos principais centros pedagógicos atuantes na região. & Tornar uma ação de caráter também corretivo. Ampliar para todas as redes de ensino, instituições, empresas e comércio. Seguir aplicação de ações propostas na \autoref{sec:educ_amb}. \\
	\rowcolor[rgb]{ .992,  .914,  .851} Coleta seletiva e triagem dos resíduos & P     & Indefinido no período de elaboração desse produto. A coleta setetiva e sua triagem evita que parte dos resíduos secos recicláveis sejam destinados aos aterros sanitários. & Seguir diretrizes da \autoref{sec:capac_tec} e \autoref{sec:metas}. Melhorar a eficiência do programa de coleta seletiva. Transmitir conscientização à população quanto a separação dos resíduos recicláveis. Contar com dispositivos de entrega dos resíduos secos, como em \gls{lev}. Estabeleber um termo contratual com algum local de destinação de resíduos recicláveis. Viabilizar, antecipadamente, um termo de renovação ou uma nova associação. Levantar um cadastro de locais passíveis a receberem resíduos recicláveis em caso de emergência. \\
	\rowcolor[rgb]{ .984,  .831,  .706} Entrega voluntária de resíduos & P     & Existente. Para óleos de cozinha, evitando a destinação incorreta. Os dois pontos de entrega ocorrem através de bombonas de 50 litros e localizam-se próximos ao Terminal Rodoviário e na Secretária de Meio Ambiente e Agricultura. & Aumentar a diversidade, capacidade e abrangência dos \gls{lev} e Ecopontos a serem instalados no município, visando aumentar o atendimento à população. Realizar programas para sensibilizar a população da importância da entrega voluntária dos resíduos produzidos. \\
	\rowcolor[rgb]{ .992,  .914,  .851} Manutenção preventiva de frota e equipamentos utilizados nos serviços de limpeza e disposição final de resíduos & P/C   & Inexistente. A existência de manutenção preventiva da frota e dos equipamentos utilizados no sistema de limpeza urbana e manejo de resíduos sólidos evita situações de paralização dos serviços. & Realizar revisão preventiva nos caminhões com vistas a evitar interrupções na prestação de serviço devido a problemas mecânicos. Possuir veículos reserva os opcões pré-estabelecidas de empréstimo a fim de garantir que a prestação dos serviços de coleta e disposição final de resíduos não seja afetada por problemas na frota e interrupção da circulação de um dos caminhões. Ter disponibilidade de mecânico para realizar as manutenções necessárias e/ou convênio com alguma oficina mecânica que preste este tipo de serviço periodicamente. \\
	\rowcolor[rgb]{ .984,  .831,  .706} Programa de monitoramento da eficiência dos serviços de coleta e limpeza publica & P/C   & Inexistente. Identifica problemáticas decorrentes das estruturas e serviços, possibilitando seus devidos diagnósticos e correções, acarretando em uma maior eficiência dos serviços. & Seguir diretrizes da \autoref{sec:proc_oper}, \autoref{sec:capac_tec}, \autoref{sec:grup_int} e\autoref{sec:metas}. \\ 
	\rowcolor[rgb]{ .992,  .914,  .851} Programa de monitoramento da eficiência da disposição final de resíduos sólidos & P/C   & Inexistente. O aterro de disposição (Tremembé) contabiliza a quantidade de resíduo recebida e aterrada, porém, o município não monitora as devidas quantidades, a fim de se obter uma maior eficiência. & Seguir diretrizes da \autoref{sec:capac_tec} e \autoref{sec:metas}. Melhorar a eficiência do programa de coleta comum. Transmitir conscientização à população quanto a separação dos resíduos comuns e orgânicos. Monitorar as quantidades de resíduo aterrada por mês, tanto como catalogá-los no \gls{snis}. Viabilizar, antecipadamente, um termo de renovação com o atual local de disposição final ou uma nova associação. Levantar um cadastro de locais passíveis a receberem resíduos comuns em caso de emergência. \\
	\rowcolor[rgb]{ .984,  .831,  .706} Programa de monitoramento de descarte de RCC & P/C   & Inexistente. Identifica problemáticas decorrentes das estruturas e serviços, possibilitando seus devidos diagnósticos e correções, acarretando em uma maior eficiência dos serviços. & Regularizar um local de destinação correta dos \gls{rcc}, através de um bota fora. Realizar o controle do \gls{rcc} e seu respectivo local de acordo com a \autoref{subsec:trans_rcc}. \\
	\rowcolor[rgb]{ .992,  .914,  .851} Programa de monitoramento de geradores de Logística Reversa & P/C   & Inexistente. Identifica problemáticas decorrentes das estruturas e serviços, possibilitando seus devidos diagnósticos e correções, acarretando em uma maior eficiência dos serviços. & Garantir a funcionalidade da logística reversa através de um monitoramento de seus geradores, levantados no Produto 3. Seguir diretrizes estabelecidas na \autoref{subsec:desc_lr}. \\
	\rowcolor[rgb]{ .984,  .831,  .706} Levantamento dos geradores sujeitos aos planos de gerenciamento de resíduos sólidos e ao estabelecimento de sistemas de logística reversa. & P     & Existente. O cadastro foi realizado para a elaboração do diagnóstico do PMGIRS. Facilita a fiscalização contribuindo para que sejam evitadas práticas incorretas e consequentemente prevenindo impactos adversos decorrentes do inadequado manejo dos resíduos sólidos. & Garantir a funcionalidade da logística reversa através de um monitoramento de seus geradores, levantados no Produto 3. Seguir diretrizes estabelecidas na \autoref{subsec:desc_lr}. \\
	\rowcolor[rgb]{ .992,  .914,  .851} Cadastro de aterros próximos para uma possível recepção dos resíduos comuns em caso de impeditivo de disposição final no local atualmente utilizado & P     & Há o conhecimento acerca dos empreendimentos existentes passíveis de atender o município, podendo ser uma alternativa em caso de necessidade, mas não há contratos emergenciais pré-estabelecidos. & Estabelecer contratos emergenciais conforme descritos em \autoref{sec:emerg_corr}. Fazer uso das diretrizes estabelecidas na \autoref{sec:sol_cons} e na \autoref{sec:grup_int}. \\
	\rowcolor[rgb]{ .984,  .831,  .706} Cadastro de empresas que prestam serviços de limpeza, coleta e disposição final de resíduos como opção de contratos emergenciais para suprir uma ausência não prevista dos serviços & P     & Há o conhecimento acerca das empresas existentes passíveis de atender o município, podendo ser uma alternativa em caso de necessidade, mas não há contratos emergenciais pré-estabelecidos. & Estabelecer contratos emergenciais conforme descritos em \autoref{sec:emerg_corr}. Fazer uso das diretrizes estabelecidas na \autoref{sec:sol_cons} e na \autoref{sec:grup_int}. \\
\end{tabular}%
}
	\label{tab:acoes_prevent}%
	\legend{Fonte: Elaborado pelos autores, adaptado do PMGIRS/Arujá (2018).}
\end{table}%
