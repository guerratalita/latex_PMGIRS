% Table generated by Excel2LaTeX from sheet 'Planilha1'
%\begin{landscape}
\begin{table}[htbp]
	\centering
	\caption{Add caption}
	\arrayrulecolor{black}
	\begin{tabular}{|P{6.145em} | P{16.575em} | P{12.355em} | P{5.645em}|}
		\toprule
		Objetivos & Metas & Ações & Prazos \\
		\midrule
		Reduzir a geração de RSU & Aumentar em 60 \% a qualidade da segregação dos RS comum & Divulgação constante & 2020-2040 \\
		\midrule
		&       & Afixar nos prédios públicos as categorias de resíduos e formas de descarte & 2022 \\
		\midrule
		& Reduzir em 30 \% o que for destinado a aterro (como RSU) & Incentivo à compostagem doméstica e coletiva & 2020-2040 \\
		\midrule
		&       & Criar pontos de recebimento de roupas e tecidos & 2022 \\
		\midrule
		&       & Melhorar a segregação na fonte de geração & 2020-2040 \\
		\midrule
		& Coleta seletiva & Recipientes para segregação correta em todos os prédios públicos & 2024 \\
		\midrule
		&       & Incentivo a programa de pontos & 2028 \\
		\midrule
		&       & Divulgação constante & 2020-2040 \\
		\midrule
		&       & Expandir os pontos de coleta de óleo de cozinha usado (centros comunitários, escolas e mercados, em cada bairro) & 2022 \\
		\midrule
		& Compostagem & Compostar, localmente, 35 \% dos orgânicos gerados no município & 2028 \\
		\midrule
		&       & Composteiras em todos os prédios públicos & 2022 \\
		\midrule
		&       & Ponto de armazenagem de matéria seca para compostagem & 2022 \\
		\midrule
		Promover gestão participativa da comunidade (destaca-se aqui a importância de valorizar o conhecimento da população local) & Criar e manter canais de comunicação abertos com ênfase na Educação Ambiental & Manter as redes sociais ativas & 2020-2040 \\
		\midrule
		&       & Promover, periodicamente, rodas de conversa com munícipes para melhoria do diálogo com a sociedade & 2020-2040 \\
		\midrule
		&       & Incentivar o engajamento contínuo da rede escolar presente no município & 2020-2040 \\
		\midrule
		&       & Priorização da educação ambiental nos currículos escolares &   \\
		\midrule
		&       & Promover, periodicamente, oficinas participativas com a sociedade & 2020-2040 \\
		\midrule
		Fomentar ações que possibilitem geração de renda via resíduos & Organizar uma associação ou cooperativa de catadores de materiais reutilizáveis ou recicláveis, respeitando as capacidades de liderança dos próprios cooperados & Dialogar com os catadores autônomos & 2022 \\
		\midrule
		&       & Consulta ao Programa Pró-Catador para obtenção de recursos para a infraestrutura & 2022 \\
		\midrule
		&       & Estudo sobre possíveis terrenos públicos onde a cooperativa/associação poderá ser instalada & 2022 \\
		\midrule
		&       & Incentivar a criação por meio de abatimento de impostos ou linhas de crédito diferenciadas & 2024 \\
		\midrule
		&       & Construção da cooperativa/associação & 2024 \\
		\midrule
		& Incentivos a empresas que utilizem, no mínimo, o princípio dos 3R & Abatimento de impostos, linhas de crédito diferenciadas & 2020-2040 \\
		\midrule
		& Incentivar a criação/instalação de empresas que façam o beneficiamento (agregar valor) aos resíduos provenientes da coleta seletiva & Abatimento de impostos, linhas de crédito diferenciadas & 2020-2040 \\
		\midrule
		Melhoria dos serviços de limpeza urbana e manejo de RS & Padronização e instalação de "lixeiras" & Definição de modelo mais ergonômico aos coletores (com porta articulada, por exemplo), duradouro e com separação para orgânico e reciclável. Preferencialmente com cobertura contra chuvas. &   \\
		\midrule
		& Otimizar as rotas dos caminhões coletores &   &   \\
		\midrule
		& Criar uma escala de manutenção preventiva dos equipamentos &   &   \\
		\midrule
		& Manter o controle quantitativo da movimentação dos resíduos &   &   \\
		\midrule
		Correção do gerenciamento dos RCC & Regularizar terreno para disposição de RCC & Interromper a disposição de RCC para o "bota-fora" municipal & 2022 \\
		\midrule
		&       & Contatar o órgão ambiental competente para verificar o procedimento & \multicolumn{1}{c|}{2020} \\
		\midrule
		& \multicolumn{1}{c|}{Criar meios de reaproveitamento de RCC} & Criação de pontos de entrega, armazenagem e troca de RCC & \multicolumn{1}{c|}{2022} \\
		\midrule
		&       & Estimular intercâmbio de materiais RCC entre os munícipes & 2020-2040 \\
		\midrule
		Logística Reversa & Incentivar a implantação da logística reversa pós-consumo - parcerias, programa de pontos &   &   \\
		\midrule
		Preencher correta e regularmente os formulários do SNIS & Criar um padrão de coleta dos dados para evitar futuras interpretações ambíguas & Verificar os dados solicitados & 2020-2040 \\
		\midrule
		&       & Manter o registro das informações em meio digital de modo acessível ao responsável na gestão corrente & 2020-2040 \\
		\midrule
		&       & Evitar estimar dados, utilizar valores concretos (exemplo: valores de contratos, pesos e medidas coletados, etc.) & 2020-2040 \\
		\midrule
		&       & Utilizar valores atuais (correntes no ano) & 2020-2040 \\
		\midrule
		& Aprimorar anualmente o preenchimento & Coletar durante o ano os dados solicitados & 2020-2040 \\
		\midrule
		&       & Verificar as dificuldades enfrentadas para o preenchimento (ausência de dados, dificuldade de coleta, etc) e montar uma estratégia para solucioná-las no ano seguinte & 2020-2040 \\
		\bottomrule
	\end{tabular}%
	\label{tab:addlabel}%
\end{table}%
%\end{landscape}