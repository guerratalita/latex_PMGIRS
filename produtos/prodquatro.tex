	\newcolumntype{P}[1]{>{\centering\arraybackslash}p{#1}} %para centralizar as colunas nas tabelas

\thispagestyle{headfootimage}

Os objetivos e metas deste produto estão alinhados com a \gls{pnrs}, \gls{pnsb}, normas das três esferas de poder e também com os \gls{ods}.

O prognóstico do sistema de manejo de resíduos sólidos foi projetado para o horizonte de 20 anos de vigência do plano (período de 2019 a 2039) As estimativas da geração de resíduos por classe foram baseadas nas taxas de geração do diagnóstico (Produto 3) e na projeção da população no período. É importante ressaltar que 

\section{Metas}

Criação de um grupo de apoio mútuo para comprimento das metas estabelecidas no plano. O grupo deverá contar com representantes do poder público, da sociedade e pessoal técnico e capacitado. Parceria com instituições para apoio técnico.

Reativação do Conselho Municipal de Meio Ambiente para implementação do plano e afins.


\section{Projeções para o horizonte de 20 anos do Plano}

%	https://www.mma.gov.br/estruturas/srhu_urbano/_arquivos/1_manual_elaborao_plano_gesto_integrada_rs_cp_125.pdf


%	https://www.mma.gov.br/cidades-sustentaveis/residuos-solidos/material-t%C3%A9cnico.html

%	http://www.pmf.sc.gov.br/sistemas/pmgirs/arquivos/PMGIRS_CADERNO_3_Prognostico_PMGIRS_Florianopolis.pdf

PGRSS para os estabelecimentos geradores de RSS
Descarte específico chapa de raio X
separação de resíduos na ubs 
recipientes rígidos para armazenagem dos resíduos na ubs

novamente, dificuldade para interpretar os dados devido a falta de dados consistentes

cadastro dos geradores de rss
controle dos pesos/volumes/massas dos geradores particulares, por classe de resíduo

rcc - mun: caracterizar grandes e pequenos geradores para facilitar a elaboração PGRCC

pequenos volumes: pev - troca; incentivo a recuperação de móveis (fonte de renda) - coleta e doa

informações de acesso livre

aumentar pontos de coleta de óleo - gerenciar retirada
\subsection{Projeção populacional: total, urbana, per capta}

A estimativa de crescimento populacional foi feita com base nos dados do Censo de 2010 do \gls{ibge} e utilizando o método geométrico para o crescimento das populações através de uma planilha disponibilizada pelo \gls{mma} (Planilha MMA)  %https://www.mma.gov.br/cidades-sustentaveis/residuos-solidos/material-t%C3%A9cnico.html).

% Table generated by Excel2LaTeX from sheet 'Planilha1'
\begin{table}[htbp]
  \centering
  \arrayrulecolor{white}
  \caption{Projeção da população do município de Monteiro Lobato para o horizonte de 20 anos do Plano}
    \begin{tabular}{c|c|c|c}
    \rowcolor[rgb]{ .969,  .588,  .275} \multicolumn{4}{c}{\textcolor[rgb]{ 1,  1,  1}{\textbf{Projeção populacional}}}\\
    \rowcolor[rgb]{ .984,  .831,  .706} \textbf{Ano} & \textbf{Pop. Urbana} & \textbf{Pop. Rural} & \textbf{Total} \\
    \rowcolor[rgb]{ .992,  .914,  .851} 2020  & 1.983  & 2.482  & 4.465 \\
    \rowcolor[rgb]{ .984,  .831,  .706} 2025  & 2.069  & 2.525  & 4.594 \\
    \rowcolor[rgb]{ .992,  .914,  .851} 2030  & 2.139  & 2.544  & 4.683 \\
    \rowcolor[rgb]{ .984,  .831,  .706} 2035  & 2.193  & 2.544  & 4.737 \\
    \rowcolor[rgb]{ .992,  .914,  .851} 2040  & 2.228  & 2.518  & 4.746 \\
    \end{tabular}%	
  \label{tab:proj_pop}%
  \legend{Fonte: Fundação Seade.}
\end{table}%


Caso a projeção populacional se confirme, Monteiro Lobato terá um acréscimo de aproximadamente XXXXX pessoas que devem contribuir para o aumento da geração de resíduos sólidos no município. Importante ressaltar que o próximo Censo do \gls{ibge} deverá ocorrer em breve e gerar dados populacionais mais próximos da realidade municipal, que podem diferir da projeção, ainda assim, a projeção é muito útil para fins de planejamento futuro.

\subsection{Projeção da geração de resíduos total e per capta (por classe)}

As projeções foram feitas pelo método geométrico para as classes de resíduos para as quais o município dispõe de dados, que são os resíduos comuns e os resíduos de serviços de saúde. 

%\input{./produtos/prodquatro/proj_residuos.tex}

As taxas per capta de geração de resíduos foram estimadas pelas médias de geração do período entre 2014 e 2017. Os dados dos resíduos de saúde foram estimados com base na quantidade de resíduos recolhida pela empresa especializada AGIT Soluções Ambientais Ltda para o período. Os resíduos de saúde são classificados em 5 categorias (seção xxxxx P3) e a coleta realizada pela empresa Soluções Ambientais Ltda ocorre para os grupo de resíduos A e E. Os outros grupos são recolhidos pela coleta *****comum e não são quantificados, de modo que não há informações detalhadas sobre eles*****.

Analisando as projeções, é possível estimar um aumento da geração de \gls{rss} que ultrapassaria, em 2023, a quantidade atualmente contratada de 1,8 tonelada anual de resíduo a ser coletada e disposta pela empresa especializada, ademais, os outros grupos de resíduos de \gls{rss} (B, C, E) recolhidos pela coleta comum também devem aumentar significando elevação custos para essa classe. O grupo de resíduos D geralmente é composto por resíduos passíveis de reciclagem e representa uma grande porção do total (75\% a 90\%) gerado em locais de serviços de saúde (MMA, 2011), logo, a correta segregação e destinação adequada desse grupo pode significar recuperação de parte dos recursos graças à reciclagem.   

\subsection{Cenários possíveis}

\subsubsection{Sem aplicação do Plano}

\subsubsection{Com aplicação do Plano}	

\section{Redução da quantidade de resíduos sólidos encaminhados para disposição final}

De acordo com o Diagnóstico Municipal Participativo (Produto 3), uma das fragilidades em relação a consistência de informações que sejam fidedignas à realidade do município é a não identificação e mensuração correta dos dados necessários para inserção na base federal \gls{snis}. A ausência de informações corretas a respeito dos serviços de saneamento pode impedir ou dificultar o conhecimento sobre a situação do município e o planejamento e execução de medidas de controle, mitigação ou solução das questões.

Os formulários de preenchimento do \gls{snis} são muitas vezes extensos porém bastante completos e as informações solicitadas são de conhecimento do município (contratos, registros de operações, folhas de pagamento, etc.).	A partir do preenchimento das informações, o \gls{snis} calcula indicadores de situação que podem ser utilizados para nortear as políticas públicas em relação aos serviços de saneamento básico, dentre eles o gerenciamento de resíduos sólidos mas esses indicadores somente serão úteis se o município detiver o controle das informações prestadas. 

Em relação aos resíduos sólidos, o município deverá ter clareza da situação da geração e movimentação dos resíduos em seu território, por exemplo: quantidade total de resíduos, quantidade por classe de resíduos, flutuação da quantidade em períodos de alta temporada turística, quantidade de resíduos recicláveis, quantidade de catadores existentes, índice de atendimento de coleta urbana e rural, situação dos caminhões, etc.

Ao se pensar em metas para a diminuição dos resíduos encaminhados para disposição final ambientalmente adequada, é comum a lembrança da política dos 3Rs relacionados aos resíduos sólidos: redução, reutilização, reciclagem. De fato, a \gls{pnrs}, em seu artigo 9º, estabelece como uma de suas diretrizes a ordem de prioridade na gestão e gerenciamento de resíduos sólidos: a não geração, a redução, reciclagem, o tratamento dos resíduos sólidos e por fim a disposição ambientalmente adequada dos rejeitos. \cite{brasil:12305} Por rejeito entende-se todos os resíduos para os quais não há mais possibilidade de tratamento e recuperação, seja por ausência de tecnologia ou por não ser economicamente viável.

Relacionando a compostagem aos \gls{ods}, verifica-se que, direta ou indiretamente a atividade pode contribuir para atingir pelo menos 5 dos objetivos:

\begin{itemize}
	\item[\gls{ods} 2] Fome zero e agricultura sustentável. Ao promover a ciclagem de nutrientes, a compostagem fornece adubo para cultivos. Boa parte do território do município de Monteiro Lobato é área rural e passível de plantios; além disso há produtores na região que podem se beneficiar da atividade ao obter adubo de qualidade a preço baixo ou nenhum, caso a atividade seja desenvolvida na propriedade. Plantios próprios podem ajudar as pessoas a ter uma alimentação mais saudável e se corretamente manejados podem contribuir para a preservação do solo, da biodiversidade local, das águas e das tradições culturais alimentares.
	\item[\gls{ods} 4] Educação de qualidade. O item 4.7 fala da educação para o desenvolvimento sustentável e estilos de vida sustentáveis. Fazer a ciclagem de nutrientes de forma local contribui para ambos os pontos.
	\item[\gls{ods} 6] [Água potável e saneamento]. Garantir a disponibilidade de água de qualidade e em quantidade significa restaurar, manter e proteger os ecossistemas que possibilitam que a água chegue nos cursos d'água. Nesse sentido enriquecer o solo das propriedades com composto para que receba plantios teria como um dos benefícios ajudar a melhorar a qualidade dos recursos hídricos da região. Além disso ao estimular a destinação correta dos resíduos, diminuindo ou mesmo acabando com os pontos de descarte incorreto de resíduos orgânicos, diminui-se a poluição decorrente desse descarte e assim também melhora a qualidade das águas da bacia. 
	\item[\gls{ods} 11]	Cidades e comunidades sustentáveis. A compostagem pode ser feita de forma privada, como solução individual, comunitária, como solução coletiva ou pelo poder público. De toda forma a diminuição do impacto poluidor e das possibilidades de soluções que a atividade propicia contribui para o desenvolvimento de um espaço mais harmonioso e sustentável.
	\item[\gls{ods} 12] Consumo e produção responsáveis. Resíduos orgânicos, muitas vezes são provenientes do desperdício de alimentos, em toda a cadeia de produção até o consumidor final. Repensar a forma de lidar com o próprio resíduo pode fazer com que as pessoas repensem o consumo que gera o resíduo.  
\end{itemize}


Para que o município consiga atingir as metas para ..... gestão, seguir os indicadores já propostos no Produto 3. Essas metas são.....



\subsection{Resíduos orgânicos}

\textbf{	O tratamento dos resíduos orgânicos pode ocorrer de diversas e para diversas finalidades}

A compostagem é uma dessas formas e de acordo com a Lei de Saneamento Básico (Lei nº 11.445) também é considerada serviço de limpeza urbana e cabe ao titular dos servições de limpeza pública articular para a implantação de sistemas de compostagem.

A compostagem de resíduos orgânicos é a decomposição, na presença de oxigênio, da matéria orgânica dos resíduos de origem animal e vegetal, que têm sua carga orgânica neutralizada e transformada em matéria rica em nutrientes, sem odor e sem potencial poluidor, que podem ser incorporados ao solo tornando-o melhor para o desenvolvimento de plantas e pode ser vinculada a diversos benefícios socioambientais além da redução de resíduos a serem dispostos:

\begin{itemize}
	\item Redução de resíduos para aterramento, consequentemente redução de custos associados e aumento de vida útil do aterro;
	\item Produção de potencial adubo devido a reciclagem de nutrientes;
	\item Uso agrícola do adubo e diminuição dos custos financeiros associados à compra de fertilizantes e diminuição dos custos ambientais inerentes à produção desses fertilizantes;
	\item Diminuição da emissão de CH4 (metano) para a atmosfera, gás de elevado potencial de efeito estufa. 
\end{itemize}

Os materiais que podem ser compostados são diversos e geralmente são resíduos orgânicos em geral (domésticos ou não), aparas de grama, resíduos de poda e capina, cinzas e etc, e em geral a montagem de uma composteira, independentemente do tamanho é de uma mistura de parte de resíduo orgânico úmido para partes de matéria seca (serragem, grama seca, lascas de madeira e etc). A atividade pode ser implantada nos municípios desde que haja controle sobre a qualidade dos processos e do produto para que não ocorra a geração de compostos contaminados ou a atividade se torne contaminante do local \cite{felipetto_conceito_2007}. A segregação na fonte é uma das etapas mais importantes do processo de compostagem pois evita que contaminantes como medicamentos, pilhas e outros sejam misturados aos resíduos compostáveis e terminem por prejudicar a qualidade do material final. Por esse motivo é importante conscientizar a população e viabilizar a correta separação dos resíduos compostáveis de outros tipos de resíduos (recicláveis, resíduos de logística reversa, resíduos de construção civil e etc) e dos rejeitos.

Como abordado no Diagnóstico Municipal (Produto 3), cerca de 34 \% dos resíduos da coleta comum no município de Monteiro Lobato são compostos por matéria orgânica que poderia ser compostada, o que diminuiria a contribuição do município tanto em relação a quantidade de resíduos que deixaria de ser aterrada e consequentemente seus próprios custos, quanto a participação na produção de \gls{gee}. Boa parte dos moradores (cerca de 50 \%) e algumas comunidades municipais já fazem compostagem nos seus espaços próprios, a exemplo do\underline{\textbf{ bairro dos Souzas.}} O poder público local, focando na gestão participativa, poderia manter um diálogo com essas pessoas e comunidades e estabelecer parcerias para troca de informações, métodos e materiais, como equipamentos e matéria seca.

Quando realizado pelos geradores em ambiente domiciliar (no caso, os munícipes) é classificada como compostagem doméstica. Essa forma de compostagem é realizada em recipientes chamados de composteiras ou minhocários, caso sejam utilizadas minhocas no processo. A vantagem da composteira doméstica é que além de diminuir o impacto e os custos inerentes à coleta e destinação desses resíduos, proporciona o tratamento local do resíduo e produz adubo que pode ser utilizado pelo próprio munícipe em seu próprio jardim ou horta. Quando bem feita, não produz odores nem atrai animais vetores de doenças.

A compostagem também pode ser realizada através de Unidades ou Usinas de Compostagem e, neste caso, é uma atividade passível de licenciamento pelo órgão ambiental competente e que necessita ter projeto, implantação e operação bem detalhados e responsáveis técnicos. 

O Ministério do Meio Ambiente disponibilizou, em 2010, o Manual para implementação de compostagem e de coleta seletiva no âmbito de consórcios públicos. De acordo com esse manual, podem ser implementados  



%caso a compostagem seja feita pelo município, será necessário ter licença? será necessário atender todos os padrões para fazer adubo? vão doar? qual o custo de implantação? pessoal? treinamento/qualificação? onde poderá ser feito?


\subsection{Coleta seletiva}



\subsection{Logística Reversa pós-consumo}



\section{Responsabilidades quanto à implementação e operacionalização do Plano}



A \gls{pnrs} estabelece o conceito de responsabilidade compartilhada pelo ciclo de vida dos produtos para a gestão e gerenciamento dos resíduos gerados nos territórios. Por responsabilidade compartilhada entende-se o "conjunto de atribuições individualizadas e encadeadas dos fabricantes, importadores, distribuidores e comerciantes, dos consumidores e dos titulares dos serviços públicos de limpeza urbana e de manejo dos resíduos sólidos, para minimizar o volume de resíduos sólidos e rejeitos gerados, bem como para reduzir os impactos causados à saúde humana e à qualidade ambiental decorrentes do ciclo de vida dos produtos" \cite{brasil:12305}.

Poder público: implementação de metas para RSU/RSS/RCC
Outros - PGRS
Poder público cobra de outros o PGRS


%	A definição das responsabilidades deve ser feita quanto à implementação e à operacionalização do Plano, incluídas as etapas dos planos de gerenciamento de resíduos a que se refere o art. 20 da Lei Federal nº 12.305/2010 a cargo do poder público. 

%	Conforme o conceito de responsabilidade compartilhada pelo ciclo de vida do produto, devem ser definidas as atribuições individualizadas e encadeadas dos fabricantes, importadores, distribuidores e comerciantes, dos consumidores e dos titulares dos serviços públicos de limpeza urbana e manejo de resíduos sólidos. 

