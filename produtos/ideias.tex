%X. programas e ações de educação ambiental que promovam , na ordem, a não geração, a redução, a reutilização e a reciclagem de resíduos sólidos; 

trazer a questão dos catadores - eles conhecem o território e sua dinâmica como ninguém

reativação do conselho do meio ambiente municipal

criação de um grupo de apoio para o cumprimento das metas e dos objetivos

%articulação entre os entes das sociedade para identificação de soluções individuais ou coletivas que por ventura já sejam realizadas e que possam ser replicadas ou adaptadas em outras comunidades e setores da sociedade.

%estabelecer o canal de comunicação e mantê-lo aberto, horizontalmente, entre os entes.

realização de oficinas participativas com líderes comunitários


cobrar educação/conscientização ambiental das empresas e comércio

%levar as medidas de educação ambiental para dentro do setor público
Definição da PNEA sobre ed. ambiental e o que ela prega

aproveitamento integral dos alimentos destinando o que sobra à compostagem

%implantação de composteiras do tipo doméstico aeróbio em todos os espaços públicos (remoção de lixeiras de orgânicos das repartições)
