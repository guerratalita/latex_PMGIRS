\thispagestyle{headfootimage}




\titleformat{\subsection}
{\Large\bfseries\scshape\raggedright}
{\thesubsection}{1em}
{}
\titleformat{\subsubsection}
{\Large\bfseries\scshape\raggedright}
{\thesubsubsection}{1em}
{}
\titlespacing*{\section}{0pt}{*2}{*2}
\titlespacing*{\subsection}{0pt}{*0.5}{*0}
\titlespacing*{\subsubsection}{1cm}{*0}{*0}

\newenvironment{subapend}{%
	\begin{adjustwidth}{0.1\textwidth}{0cm}
	}{
	\end{adjustwidth}
}
\newenvironment{subsubapend}{%
	\begin{adjustwidth}{0.1\textwidth}{0cm}
	}{
	\end{adjustwidth}
}

\chapter{Classificação de exigências legais e normativas}

A seguir, as exigências legais aplicáveis à gestão municipal de resíduos sólidos em Monteiro Lobato serão classificadas em temas e em esfera de governo, segundo seu conteúdo e/ou a descrição da lei.

\section{Resíduos Sólidos Urbanos}

\begin{subapend}
	\subsection{Federal}
	\begin{subsubapend}
		\item \subsubsection{Lei 11.445/2007}	
			Sobre saneamento básico (possui conteúdo sobre RSU).
		 \subsubsection{NBR 1.299/1993}	
			Sobre a coleta, varrição e acondicionamento de resíduos sólidos urbanos – Terminologia.
		 \subsubsection{NBR 8.419/1996}
			Sobre procedimentos para a apresentação de projetos de aterro sanitário de resíduos sólidos urbanos.
		 \subsubsection{NBR 12.980/1993}
			Esta Norma define os termos utilizados na coleta, varrição e acondicionamento de resíduos sólidos urbanos.
		 \subsubsection{NBR 15.911/2011}
			Sobre o contentor móvel de plásticoParte 2: Contentor de duas rodas, com capacidade de 120 L, 240 L e 360 L, destinado à coleta de resíduos sólidos urbanos (RSU) e de saúde (RSS) por coletor compactador.
	\end{subsubapend}
\end{subapend}


\begin{subapend}
	\subsection{Estadual}
	\begin{subsubapend}
		\item \subsubsection{Constituição Estadual}
			Estabelece políticas, ações e deveres de saneamento básico.
		 \subsubsection{Lei complementar nº 1.025/200}
			Transforma a Comissão de Serviços Públicos de Energia – CSPE em Agência Reguladora de Saneamento e Energia do Estado de São Paulo – ARSESP, dispõe sobre os serviços públicos de saneamento básico e de gás canalizado no Estado, e dá outras providências.
		 \subsubsection{Lei nº 7.750/1992}
			Sobre a política estadual de saneamento.
		 \subsubsection{Lei nº 10.763/2001}
			Dispõe sobre medidas a serem adotadas na prevenção e controle às inundações. Obs: Há menções do lixo urbano como uma das principais causas de inundações.
		 \subsubsection{Lei 10.888/2001}
			Sobre o descarte final de produtos potencialmente perigosos do resíduo urbano que contenham metais pesados.
		 \subsubsection{Decreto 55.565/2010}
			Sobre a prestação de serviços públicos de saneamento básico relativos à limpeza urbana e ao manejo de resíduos sólidos urbanos no estado de São Paulo e dá providências correlatas.
		 \subsubsection{Norma Cetesb P4.241 (Sem Data)}
			Norma para apresentação de projetos de aterros sanitários de resíduos urbanos.
		 \subsubsection{Resolução SSE/SMA 49/200}
			Cria Grupo de Trabalho para propor um programa estadual de aproveitamento energético de resíduos sólidos urbanos e outros rejeitos da atividade econômica.
	\end{subsubapend}
\end{subapend}

\begin{subapend}
	\subsection{Municipal}
	\begin{subsubapend}
		\item \subsubsection{Lei Orgânica do Município de Monteiro Lobato. Promulgada em 1990 e atualizada em 2007.}
		Art. 98: Estabelece procedimentos para a implantação de Planos de obras e serviços municipais.
		\subsubsection{Lei 1.296/05}
		Locação imóvel destinado ao depósito de Merenda Escolar
		\subsubsection{Lei 1.350/07}
		Locação imóvel destinado ao Depósito de Merenda Escola. 
		\subsubsection{Lei 1.417/09}
		Locação de imóvel destinado a marcenaria e depósito do Setor de Serviços Urbanos.
		\subsubsection{Lei 1.447/09}
		Autoriza a administração municipal a podar, extrair ou Substituir Árvores condenadas ou em risco de queda, defronte a imóveis particulares, sem solicitação ou autorização do proprietário.
		\subsubsection{Decreto 99/1974}
		O município não cobrará taxas por serviços de limpeza em áreas urbanas durante período especificado. Importante para saber que, historicamente, o município não possui tradição de onerar o munícipe pelos serviços prestados.
		\subsubsection{Decreto 968/2006}
		Define os valores de créditos suplementares para secretarias, fundo municipal de saúde, fundo municipal de assistência social, serviços municipais urbanos e serviços de estrada de rodagem.
	\end{subsubapend}
\end{subapend}

\section{Resíduos de Serviços de Saneamento}

\begin{subapend}
	\subsection{Federal}
	\begin{subsubapend}
		\item \subsubsection{Resolução 375/2006}
		Define critérios e procedimentos, para o uso agrícola de lodos de esgoto gerados em estações de tratamento de esgoto sanitário e seus produtos derivados, e dá outras providências. Retificada pela Resolução nº 380 de 31 de outubro de 2006.
		\subsubsection{Resolução Conama 380/200}
		Retifica a Resolução CONAMA nº 375 de 29 de agosto de 2006 - Define critérios e procedimentos, para o uso agrícola de lodos de esgoto gerados em estações de tratamento de esgoto sanitário e seus produtos derivados, e dá outras providências.
		\subsubsection{Resolução Conama 410/2009}
		Prorroga o prazo para complementação das condições e padrões de lançamento de efluentes, previsto no art. 44 da Resolução nº 357, de 17 de março de 2005, e no Art. 3º da Resolução nº 397, de 03 de abril de 2008.
		\subsubsection{NBR 7.166/1992}
		Conexão internacional de descarga de resíduos sanitários - Formato e dimensões.
		\subsubsection{Lei nº 2.627/1954}
		Sobre a criação do Departamento de águas e esgoto do estado de São Paulo
	\end{subsubapend}
\end{subapend}

\begin{subapend}
	\subsection{Municipal}
	\begin{subsubapend}
		\item \subsubsection{Lei 864/91}
		Firma convênio com a Companhia de Saneamento Básico do Estado de São Paulo - SABESP, para obras de implantação do sistema de coleta, tratamento e disposição final de esgotos sanitários a ser executado no Município.
	\end{subsubapend}
\end{subapend}



\section{Resíduos de Construção Civil}
\subsection{Federal}
\begin{subapend}
	\begin{subsubapend}
		\item \subsubsection{Resolução Conama 307/2002}
		Sobre gerenciamento de RCC, alterada pelas resoluções 348/2004, 431/2011, 448/2012, 469/2015.
		\subsubsection{NBR 15.112/2004}
		Sobre os resíduos da construção civil e resíduos volumosos; ATTs; diretrizes para projeto, implantação e operação.
		\subsubsection{NBR 15.113/2004}
		Sobre RCC e inertes - aterros.
		\subsubsection{NBR 15.114/2004}
		Sobre RCC - áreas de reciclagem.
		\subsubsection{NBR 15.115/2004}
		Sobre RCC e agregados reciclados.
		\subsubsection{NBR 15.116/2004}
		Sobre RCC e uso de agregados em construções e pavimentações.
		\subsubsection{NR 18}
		Sobre as Condições e Meio Ambiente de Trabalho na Indústria da Construção. Possui conteúdo sobre exigências e procedimentos para armazenamento de entulho.
	\end{subsubapend}
\end{subapend}


\begin{subapend}
	\subsection{Estadual}
	\begin{subsubapend}
		\item \subsubsection{Lei nº 119/1973}
		Sobre a constituição da Sabesp.
		\subsubsection{Resolução da SMA 81/2014}
		Estabelece diretrizes para implementação do Módulo Construção Civil do Sistema Estadual de Gerenciamento Online de Resíduos Sólidos – SIGOR, e dá providências correlatas.
		\subsection{Municipal}
		\subsubsection{Lei Orgânica do Município de Monteiro Lobato. Promulgada em 1990 e atualizada em 2007.}
		\subsubsection{Art. 98}
		Estabelece procedimentos para a implantação de Planos de obras e serviços municipais.
		\subsubsection{Lei 865/91}
		Dispõe sobre doação de materiais de construção a famílias de baixa renda.
		\subsubsection{Lei 1.442/09}
		Dispõe sobre Estudo e Relatório de Impacto Ambiental nos projetos de edificações.
		\subsubsection{Lei 1.541/13}
		Sobre a obrigatoriedade do uso de tapumes de folhas ou chapas de ferro ou alumínio em obras de construções ou reformas realizadas pelos Poderes Executivo e Legislativo e pelos órgãos estaduais.
	\end{subsubapend}
\end{subapend}



\section{Resíduos Volumosos}
\begin{subapend}
	\subsection{Federal}
	\begin{subsubapend}
		\item \subsubsection{NBR 15.112/2004}
		Resíduos da construção civil e resíduos volumosos; ATTs; diretrizes para projeto, implantação e operação.
	\end{subsubapend}
\end{subapend}

\begin{subapend}
	\subsection{Municipal}
	\begin{subsubapend}
		\item \subsubsection{Lei 1.447/09}
		Autoriza a administração Municipal a Podar,Extrair ou Substituir Árvores condenadas ou em risco de queda, defronte a imóveis particulares, sem solicitação ou autorização do proprietário.
	\end{subsubapend}
\end{subapend}



\section{Resíduos de Serviço de Saúde}
\begin{subapend}
\subsection{Federal}	
	\begin{subsubapend}
		
		\item \subsubsection{Lei 9782/1999}
		Dispõe sobre os resíduos sob responsabilidade da Vigilância Sanitária. Incumbe à Agência, respeitada a legislação em vigor, regulamentar, controlar e fiscalizar os produtos e serviços que envolvam risco à saúde pública como alimentos, inclusive bebidas, águas envasadas, seus insumos, suas embalagens, aditivos alimentares, limites de contaminantes orgânicos, resíduos de agrotóxicos e de medicamentos veterinários.
		\subsubsection{Resolução Conama 06/1991}
		Dispõe sobre tratamento de RSS, resíduos de aeroportos e portos - alterada posteriormente.
		\subsubsection{Resolução Conama 358/2005}
		Sobre tratamento e disposição final de RSS. Revoga as resoluções 5/93 e 283/2001.
		\subsubsection{Resolução RDC ANVISA 305/2002}
		Sobre tratamento de RSS e materiais descartados ou acondicionados.
		\subsubsection{Resolução RDC ANVISA 306/2004}
		Sobre o gerenciamento de RSS - revoga a RDC 33/2003.
		\subsubsection{NBR 12.807/1993}
		Sobre os resíduos de serviços de saúde – Terminologia.
		\subsubsection{NBR 12.808/2016}
		Sobre os resíduos de serviço de saúde – Classificação.
		\subsubsection{NBR 12.809/2013}
		Esta Norma estabelece os procedimentos necessários ao gerenciamento intra estabelecimento de resíduos de serviços de saúde os quais, por seus riscos biológicos e químicos, exigem formas de manejo específicos, a fim de garantir condições de higiene, segurança e proteção à saúde e ao meio ambiente.
		\subsubsection{NBR 12.810/2016.}
		Sobre o gerenciamento de RSS fora do ambiente gerador.
		\subsubsection{NBR 14.652/2013}
		Sobre coleta e transporte de RSS.
		\subsubsection{NBR 15.911/2011}
		Contentor móvel de plástico. Parte 2: Contentor de duas rodas, com capacidade de 120 L, 240 L e 360 L, destinado à coleta de resíduos sólidos urbanos (RSU) e de saúde (RSS) por coletor compactador.
	\end{subsubapend}
\end{subapend}

\begin{subapend}
	\subsection{Estadual}
	\begin{subsubapend}
		\item \subsubsection{Decisão Cetesb nº. 3-E/2004}
		Homologa a Norma Técnica P4.262 - Gerenciamento de Resíduos Químicos Provenientes de Estabelecimentos de Serviços de Saúde - Procedimento (dezembro/2003).
		\subsubsection{Decisão Cetesb nº. 224/2007/E}
		Dispõe sobre a homologação da revisão da Norma Técnica P.4.262 - Gerenciamento de Resíduos Químicos provenientes de Estabelecimentos de Serviços de Saúde - Procedimento - agosto/2007 - e dá outras providências.
		\subsubsection{Norma Cetesb P4.262/2007}
		Gerenciamento de resíduos químicos provenientes de estabelecimentos de serviço de saúde - procedimento.
		\subsubsection{Norma Cetesb E15.010/2011}
		Sistemas de tratamento térmico sem combustão de resíduos de serviços de saúde contaminados biologicamente: procedimento.
		\subsubsection{Norma Cetesb E15.011/2007}
		Sistema de Incineração de Resíduos de Serviços de Saúde - Procedimento.
		\subsubsection{Portaria da Secretaria Estadual de Meio Ambiente  CVS nº 21/2008}
		Normas para gerenciamento de RSS.
		\subsubsection{SS/SMA/SJDC-SP 1/2004}
		Estabelece classificação, as diretrizes básicas e o regulamento técnico sobre Resíduos de Serviços de Saúde Animal - R.S.S.A
		\subsubsection{Resolução da SMA 22/2007}
		Estabelece que os resíduos citados pela Conama 358/2005 devem ter estabelecimentos de tratamento licenciados pela Cetesb.
		\subsubsection{Resolução da SMA 33/2005}
		Dispõe sobre procedimentos para o gerenciamento e licenciamento ambiental de sistemas de tratamento e disposição final de resíduos de serviços de saúde humana e animal no Estado de São Paulo. Revoga a 31/2003.
		\subsubsection{Resolução da SMA 103/2012}
		Dispõe sobre a fiscalização do gerenciamento de resíduos de serviços de saúde.
	\end{subsubapend}
\end{subapend}


\begin{subapend}
	\subsection{Municipal}
	\begin{subsubapend}
		\item \subsubsection{Lei Orgânica do Município de Monteiro Lobato. Promulgada em 1990 e atualizada em 2007.}
		\subsubsection{Art. 98}
		Estabelece procedimentos para a implantação de Planos de obras e serviços municipais.
		\subsubsection{Decreto 968/2006}
		Define os valores de créditos suplementares para secretarias, fundo municipal de saúde, fundo municipal de assistência social, serviços municipais urbanos e serviços de estrada de rodagem.
	\end{subsubapend}
\end{subapend}


 
\section{Resíduos Agrossilvopastoris}
\begin{subapend}
	\subsection{Federal}
	\begin{subsubapend}
		\item \subsubsection{Lei 7802/1989}
		Sobre a pesquisa, a experimentação, a produção, a embalagem e rotulagem, o transporte, o armazenamento, a comercialização, a propaganda comercial, a utilização, a importação, a exportação, o destino final dos resíduos e embalagens, o registro, a classificação, o controle, a inspeção e a fiscalização de agrotóxicos, seus componentes e afins, e dá outras providências.
		\subsubsection{Lei 9782/1999}
		Dispõe sobre os resíduos sob responsabilidade da Vigilância Sanitária. Incumbe à Agência, respeitada a legislação em vigor, regulamentar, controlar e fiscalizar os produtos e serviços que envolvam risco à saúde pública como alimentos, inclusive bebidas, águas envasadas, seus insumos, suas embalagens, aditivos alimentares, limites de contaminantes orgânicos, resíduos de agrotóxicos e de medicamentos veterinários.
		\subsubsection{Lei 9974/2000}
		Altera a Lei no 7.802/1989, que dispõe sobre a pesquisa, a experimentação, a produção, a embalagem e rotulagem, o transporte, o armazenamento, a comercialização, a propaganda comercial, a utilização, a importação, a exportação, o destino final dos resíduos e embalagens, o registro, a classificação, o controle, a inspeção e a fiscalização de agrotóxicos, seus componentes e afins, e dá outras providências.
		\subsubsection{Decreto 4074/2002}
		Regulamenta a lei 7802/1989 e a lei 3550/2000.
		\subsubsection{NBR 13.227/2017}
		Sobre agrotóxicos e afins - Determinação de resíduo não volátil
		\subsubsection{NBR 13.237/2017}
		Esta Norma especifica um método de ensaio para determinação do resíduo por peneiramento úmido de produtos agrotóxicos e afins. 
		\subsubsection{NBR 14.719/2001}
		Estabelece os procedimentos para a destinação final das embalagens rígidas, usadas, vazias, adequadamente lavadas de acordo com a NBR 13968, que contiveram formulações de agrotóxicos miscíveis ou dispersíveis em água.
		\subsection{Estadual}
		\subsubsection{Lei nº 10.547/2000}
		Define procedimentos, proibições, estabelece regras de execução e medidas de precaução a serem obedecidas quando do emprego do fogo em práticas agrícolas, pastoris e florestais. Obs: há sugestões de uso de resíduos agrícola nessas práticas.
		\subsubsection{Decisão Cetesb nº. 88/2012}
		Dispõe sobre a prorrogação do prazo fixado para que as pessoas físicas e/ou jurídicas que possuam estoques de agrotóxicos obsoletos, em especial os considerados POP's, declarem a situação de seu armazenamento e acondicionamento, com vistas à elaboração de projeto para a eliminação desses resíduos no Estado de São Paulo, e dá outras providências.
		\subsubsection{Decisão Cetesb nº. 273/2010}
		Dispõe sobre a Homologação da Norma Técnica de Efluentes e Lodos Fluidos de Indústrias Cítricas - Critérios e Procedimentos para aplicação no solo agrícola.
		\subsubsection{Decisão Cetesb nº. 388/2010}
		Aprova premissas e diretrizes para a aplicação de resíduos e efluentes em solo agrícola no Estado de São Paulo.
		\subsubsection{Norma Cetesb P4.231/2006}
		Sobre a vinhaça - Critérios e Procedimentos para Aplicação no Solo Agrícola Norma Cetesb P4.262 (2004) Dispõe sobre procedimentos para utilização de resíduos em fornos de produção clinquer (processo E/341/2003) – dezembro de 2003.
		\subsubsection{Resolução da SMA 50/2007}
		Define as diretrizes para a adequação ambiental de imóveis rurais com vistas à participação no Projeto Mina D’Água. Obs: possui exigências em relação aos resíduos sólidos.
	\end{subsubapend}
\end{subapend}

\begin{subapend}
\subsection{Municipal}	
	\begin{subsubapend}
		\item \subsubsection{Lei Orgânica (Resolução 1/2007)}
		Cap. XVIII - o-) “ao uso e armazenamento dos agrotóxicos, seus componentes e afins, bem como, a coleta e ao controle diferenciado do lixo produzido por estes produtos”;   
	\end{subsubapend}
\end{subapend}

\section{Resíduos de Logística Reversa}

\begin{subapend}
	\subsection{Federal}
	\begin{subsubapend}
		\item \subsubsection{Lei 7802/1989}
		Sobre a pesquisa, a experimentação, a produção, a embalagem e rotulagem, o transporte, o armazenamento, a comercialização, a propaganda comercial, a utilização, a importação, a exportação, o destino final dos resíduos e embalagens, o registro, a classificação, o controle, a inspeção e a fiscalização de agrotóxicos, seus componentes e afins, e dá outras providências.
		\subsubsection{Lei 9177/2017}
		Regulamenta o art. 33 da Lei nº 12.305, de 2 de agosto de 2010, que institui a Política Nacional de Resíduos Sólidos, e complementa os art. 16 e art. 17 do Decreto nº 7.404, de 23 de dezembro de 2010 e dá outras providências.
		\subsubsection{Lei 9974/2000}
		Altera a Lei no 7.802/1989, que dispõe sobre a pesquisa, a experimentação, a produção, a embalagem e rotulagem, o transporte, o armazenamento, a comercialização, a propaganda comercial, a utilização, a importação, a exportação, o destino final dos resíduos e embalagens, o registro, a classificação, o controle, a inspeção e a fiscalização de agrotóxicos, seus componentes e afins, e dá outras providências.
		\subsubsection{Decreto 4074/2002}
		Regulamenta a lei 7802/1989 e a lei 3550/2000.
		\subsubsection{Resolução Conama 362/2005}
		Sobre coleta e destinação de óleo usado ou contaminado - revoga a Resolução 9/1993 e foi alterada pela Resolução 450/2012.
		\subsubsection{Resolução Conama 362/2005.}
		Dispõe sobre o recolhimento, coleta e destinação final de óleo lubrificante usado ou contaminado.
		\subsubsection{Resolução Conama 401/2008}
		Sobre gerenciamento e limites de metais em pilhas e baterias que contêm chumbo, mercúrio e cádmio - revoga a 257/1999.
		\subsubsection{Resolução Conama 416/2009}
		Dispõe sobre a prevenção à degradação ambiental causada por pneus inservíveis e sua destinação ambientalmente adequada (Revoga a resolução 258/1999).
		\subsubsection{Resolução Conama 424/2010}
		Altera um parágrafo da Resolução 401/2008, sobre importação de pilhas e baterias.
		\subsubsection{Resolução Conama 450/2012}
		Altera os arts. 9º, 16, 19, 20, 21 e 22, e acrescenta o art. 24-A à Resolução nº 362/2005, que dispõe sobre recolhimento, coleta e destinação final de óleo lubrificante usado ou contaminado.
		\subsubsection{Resolução Conama 465/2014}
		Sobre os requisitos e critérios técnicos mínimos necessários para o licenciamento ambiental de estabelecimentos destinados ao recebimento de embalagens de agrotóxicos e afins, vazias ou contendo resíduos. Revoga a Resolução nº 334/2003.
		\subsubsection{NBR 13.227/2017}
		Agrotóxicos e afins - Determinação de resíduo não volátil
		\subsubsection{NBR 13.237/2017}
		Esta Norma especifica um método de ensaio para determinação do resíduo por peneiramento úmido de produtos agrotóxicos e afins. 
		\subsubsection{NBR 14.719/2001}
		Estabelece os procedimentos para a destinação final das embalagens rígidas, usadas, vazias, adequadamente lavadas de acordo com a NBR 13968, que contiveram formulações de agrotóxicos miscíveis ou dispersíveis em água.
		\subsubsection{NBR 15.833/2010}
		Esta Norma prescreve os procedimentos para o transporte, armazenamento e desmonte com reutilização, recuperação dos materiais recicláveis e destinação final de resíduos dos aparelhos de refrigeração.
		\subsubsection{NBR 16.156/2013}
		Esta Norma estabelece requisitos para proteção ao meio ambiente e para o controle dos riscos de segurança e saúde no trabalho na atividade de manufatura reversa de resíduos eletroeletrônicos.
		\subsubsection{IN do IBAMA 01/2010}
		Procedimentos do Ibama e sobre fabricação, importação, coleta e destinação final de pneus.
		\subsubsection{IN do IBAMA 08/2012}
		Procedimentos e destinação final de pilhas e baterias.
	\end{subsubapend}
\end{subapend}

\begin{subapend}
	\subsection{Estadual}
	\begin{subsubapend}
		\item \subsubsection{Lei nº 12.288/2006}
		Dispõe sobre a eliminação controlada dos PCBs e dos seus resíduos, a descontaminação de transformadores, capacitores e demais equipamentos elétricos que contenham PCBs, e dá providências correlatas.
		\subsubsection{Lei 13.576/2009}
		Sobre a destinação, reciclagem e gerenciamento do lixo tecnológico.
		\subsubsection{Lei nº 14.186/2010}
		Dispõe sobre a coleta, o recolhimento e o destino final das embalagens plásticas de óleos lubrificantes, e dá outras providências correlatas.
		\subsubsection{Lei nº 15.276/2014}
		Dispõe sobre a destinação de veículos em fim de vida útil e dá outras providências.
		\subsubsection{Decreto nº 60.150/2014}
		Regulamenta a Lei nº 15.276, de 2014, que dispõe sobre a destinação de veículos em fim de vida útil.
		\subsubsection{Decisão Cetesb nº. 88/2012}
		Dispõe sobre a prorrogação do prazo fixado para que as pessoas físicas e/ou jurídicas que possuam estoques de agrotóxicos obsoletos, em especial os considerados POP's, declarem a situação de seu armazenamento e acondicionamento, com vistas à elaboração de projeto para a eliminação desses resíduos no Estado de São Paulo, e dá outras providências.
		\subsubsection{Resolução da SMA - SP 11/2012}
		Trata dos programas de responsabilidade pós-consumo no setor da telefonia móvel celular.
		\subsubsection{Resolução da SMA 115/2013}
		Trata do estabelecimento de programas de responsabilidade pós-consumo para os medicamentos domiciliares, vencidos ou em desuso.
	\end{subsubapend}
\end{subapend}	


\section{Resíduos Industriais}
\begin{subapend}
	\subsection{Federal}
	\begin{subsubapend}
		\item \subsubsection{NR 25}
		Sobre os Resíduos Industriais.
	\end{subsubapend}
\end{subapend}

\begin{subapend}
	\subsection{Estadual}
	\begin{subsubapend}
		\item \subsubsection{Norma Cetesb L5.510/1982}
		Lixiviação de resíduos industriais: Método de Ensaio.
		\subsubsection{Norma Cetesb L10.101/1988}
		Sobre os resíduos sólidos industriais – tratamento no solo: Procedimento.
	\end{subsubapend}
\end{subapend}

\section{Resíduos de Serviços de Transporte}
\begin{subapend}
	\subsection{Federal}
	\begin{subsubapend}
		\item \subsubsection{Resolução Conama 02/1991}
		Sobre os procedimentos para o tratamento de cargas deterioradas e sobre a competência pela solução e pelos custos de avaliação, monitoramento, controle e gerenciamento dos resíduos gerados pelas cargas.
		\subsubsection{Resolução Conama 06/1991}
		Dispões sobre tratamento de RSS, resíduos de aeroportos e portos - alterada posteriormente.
		\subsubsection{Resolução RDC ANVISA 56/2008}
		Dispõe sobre o Regulamento Técnico de Boas Práticas Sanitárias no Gerenciamento de Resíduos Sólidos nas áreas de Portos, Aeroportos, Passagens de Fronteiras e Re­cintos Alfandegados.
	\end{subsubapend}
\end{subapend}

\section{Resíduos de Mineração}

\begin{subapend}
	\subsection{Federal}
	\begin{subsubapend}
		\item \subsubsection{NBR 13.029/2017}
		Esta Norma especifica os requisItos mínimos para a elaboração e apresentação de projeto de pilha para disposição de estéril gerado por lavra de mina a céu aberto ou de mina subterrânea, visando atender às condições de segurança, operacionalidade, economia e desativação, minimizando os impactos ao meio ambiente.
		\subsubsection{NR 22}
		Sobre a Segurança e Saúde Ocupacional na Mineração. Aplicável no monitoramento das ações relacionadas aos resíduos de mineração.
	\end{subsubapend}
\end{subapend}


\begin{subapend}
	\subsection{Municipal}
	\begin{subsubapend}
		\item \subsubsection{Lei Orgânica Art. 179}
		Exige licença para atividades de mineração e obtenção de consolidados rochosos do solo, particulados ou não.
	\end{subsubapend}
\end{subapend}

\section{Resíduos Perigosos}

\begin{subapend}
	\subsection{Federal}
	\begin{subsubapend}
		\item \subsubsection{Lei 9966/2000}
		Dispõe sobre a prevenção, o controle e a fiscalização da poluição causada por lançamento de óleo e outras substâncias nocivas ou perigosas em águas sob jurisdição nacional e dá outras providências.
		\subsubsection{Decreto 875/1993}
		Sobre a Convenção de Brasileira, de 1989, sobre o controle de movimentos transfronteiriços de resíduos perigosos e seu depósito.
		\subsubsection{Decreto no 2.063/1983}
		Estabelece multas por infrações ligadas ao transporte de produtos perigosos.
		\subsubsection{Decreto 4.136/2002}
		Regulamenta a lei 9966/2000.
		\subsubsection{Portaria Ministerial 261/1989}
		Sobre o transporte rodoviário de produtos perigosos
		\subsubsection{ANTT: Resolução 420/2004}
		Sobre as instruções complementares do transporte terrestre de produtos perigosos e substituiu Portarias publicadas pela ANTT entre 1989 e 2001. A Resolução foi alterada pela Resolução ANTT no 701/2004
		\subsubsection{Resolução Conama 023/1996}
		Regulamenta a importação e uso de resíduos perigosos. Revoga a Resolução nº 37, de 1994. Alterada pelas Resoluções nº 235, de 1998, e nº 244, de 1998. Revogada pela Resolução nº 452, de 2012.
		\subsubsection{Resolução Conama 228/1997}
		Dispõe sobre a importação de desperdícios e resíduos de acumuladores elétricos de chumbo.
		\subsubsection{Resolução Conama 252/2012}
		Sobre os procedimentos de controle da importação de resíduos conforme as normas adotadas pela Convenção da Basileia sobre o controle de movimentos transfronteiriços de resíduos perigosos e seu depósito. Revogou todas as Resoluções do CONAMA as quais tratavam da matéria até então.
		\subsubsection{Resolução Conama 420/2009}
		Dispõe sobre critérios e valores orientadores de qualidade do solo quanto à presença de substâncias químicas e estabelece diretrizes para o gerenciamento ambiental de áreas contaminadas por essas substâncias em decorrência de atividades antrópicas.
		\subsubsection{Resolução 452/2012}
		Sobre os procedimentos de controle da importação de resíduos, conforme as normas adotadas pela Convenção da Basiléia sobre o Controle de Movimentos Transfronteiriços de Resíduos Perigosos e seu Depósito - Revoga as Resoluções nº 08/1991, nº 23/1996, nº 235/1998 e nº 244/1998.
		\subsubsection{CP ANVISA 32/2004}
		Sobre a simbologia para resíduos perigosos.
		\subsubsection{NBR 7.500/2017}
		Sobre identificação e simbologias de resíduos perigosos.
		\subsubsection{NBR 8.418/1984}
		Sobre a presentação de projetos de aterros de resíduos industriais perigosos - Procedimento.
		\subsubsection{NBR 9.735/2006}
		Conjunto de equipamentos para emergências no transporte terrestre de produtos perigosos.
		\subsubsection{NBR 10.157/1987}
		Aterros de resíduos perigosos - Critérios para projeto, construção e operação – Procedimento.
		\subsubsection{NBR 11.175/1990}
		Sobre a incineração de resíduos sólidos perigosos - Padrões de desempenho – Procedimento.
		\subsubsection{NBR 13.853/1997}
		Sobre gerenciamento de resíduos descartáveis perfurantes e cortantes.
		\subsubsection{NBR 14.725/2014}
		Sobre os riscos à saúde e ao meio ambiente, que substâncias químicas podem apresentar.
		\subsubsection{NBR 16.725/2014}
		Sobre resíduo químico - Informações sobre segurança, saúde e meio ambiente - Ficha com dados de segurança de resíduos químicos (FDSR) e rotulagem.
		\subsubsection{NR 20}
		Sobre a Segurança e Saúde no Trabalho com Inflamáveis e Combustíveis. Aplicável sempre que esse tipo de material for manuseado.
		\subsubsection{NR 23}
		Sobre a Proteção Contra Incêndios. Aplicável sempre que houver risco de incêndio em função da característica de alguns resíduos sólidos.
		\subsubsection{IN do IBAMA 05/2012}
		Sobre transporte de resíduos perigosos.
		\subsubsection{IN do IBAMA 01/2013}
		Refere-se ao cadastro Nacional de Operadores de Resíduos Perigosos e prestação de informações sobre resíduos sólidos.
	\end{subsubapend}
\end{subapend}

\begin{subapend}
	\subsection{Estadual}
	\begin{subsubapend}
		\item \subsubsection{Lei 10.888/2001}
		Sobre o descarte final de produtos potencialmente perigosos do resíduo urbano que contenham metais pesados.
		\subsubsection{Lei nº 12.684/2007}
		Proíbe o uso, no Estado de São Paulo de produtos, materiais ou artefatos que contenham quaisquer tipos de amianto ou asbesto ou outros minerais que, acidentalmente, tenham fibras de amianto na sua composição. Alterada pela lei nº 16.048/2015.
		\subsubsection{Lei nº 15.303/2014}
		Institui o Programa Estadual de Incentivo ao uso de matérias–primas e insumos derivados de materiais reciclados provenientes da indústria petroquímica.
		\subsubsection{Lei nº 15.313/2014}
		Dispõe sobre a proibição do uso, armazenamento e reparo de instrumentos de medição como esfigmomanômetros e termômetros contendo mercúrio e dá outras providências.
		\subsubsection{Decreto nº 45.643/2001}
		Dispõe sobre a obrigatoriedade da aquisição pela Administração Pública Estadual de lâmpadas de maior eficiência energética e menor teor de mercúrio, por tipo e potência, e dá providências correlatas.
		\subsubsection{Decisão Cetesb nº. 3-E/2004}
		Homologa a Norma Técnica P4.262 - Gerenciamento de Resíduos Químicos Provenientes de Estabelecimentos de Serviços de Saúde - Procedimento (dezembro/2003).
		\subsubsection{Decisão Cetesb nº. 27/2008}
		Dispõe sobre a aprovação do Procedimento para Utilização de Resíduos Perigosos da Indústria Têxtil em Caldeiras, no Estado de São Paulo.
		\subsubsection{Decisão Cetesb nº. 224/2007/E}
		Dispõe sobre a homologação da revisão da Norma Técnica P.4.262 - Gerenciamento de Resíduos Químicos provenientes de Estabelecimentos de Serviços de Saúde - Procedimento - agosto/2007 - e dá outras providências.
		\subsubsection{Decisão Cetesb nº. 145/2010}
		Dispõe sobre a aprovação do Procedimento de gerenciamento de resíduos de aparas de couro e de pó de rebaixadeira oriundos do curtimento ao cromo.
		\subsubsection{Decisão Cetesb nº. 152/2007}
		Dispõe sobre procedimentos para gerenciamento de areia de fundição.
		\subsubsection{Decisão Cetesb nº. 263/2009}
		Dispõe sobre a aprovação do Roteiro para Execução de Investigação Detalhada e Elaboração de Plano de Intervenção em Postos e Sistemas Retalhistas de Combustíveis.
		\subsubsection{Norma Cetesb O1.012/1985}
		Sobre o projeto e operação de aterros industriais para resíduos perigosos: Procedimento.
		\subsubsection{Norma Cetesb P4.262/2007}
		Sobre o gerenciamento de resíduos químicos provenientes de estabelecimentos de serviço de saúde - procedimento.
		\subsubsection{Norma Cetesb E15.010/2011}
		Sistemas de tratamento térmico sem combustão de resíduos de serviços de saúde contaminados biologicamente: procedimento.
		\subsubsection{Resolução da SMA 38/2011}
		Estabelece a relação de produtos geradores de resíduos de significativo impacto ambiental, para fins do disposto no artigo 19, do Decreto Estadual nº 54645/2009, que regulamenta a Lei Estadual nº 12300/2006, e dá providências correlatas. Obs: contém exigências para o comércio de produtos farmacêuticos, cosméticos e de limpeza doméstica.
		\subsubsection{Resolução da SMA 115/2013}
		Trata do estabelecimento de programas de responsabilidade pós-consumo para os medicamentos domiciliares, vencidos ou em desuso.
	\end{subsubapend}
\end{subapend}

\section{Reciclagem, Reúso ou Reaproveitamento}

\begin{subapend}
	\subsection{Federal}
	\begin{subsubapend}
		\item \subsubsection{Decreto 5.940/2006}
		Institui a separação dos resíduos recicláveis descartados pelos órgãos e entidades da administração pública federal direta e indireta, na fonte geradora, e a sua destinação às cooperativas.
		\subsubsection{NBR 15.114/2004}
		Sobre RCC - áreas de reciclagem.
		\subsubsection{NBR 15.115/2004}
		Sobre RCC e agregados reciclados.
		\subsubsection{NBR 15.116/2004}
		Sobre RCC e uso de agregados em construções e pavimentações.
		\subsubsection{Lei nº 14.470/2011}
		Dispõe sobre a separação dos resíduos recicláveis descartados pelos órgãos e entidades da administração pública estadual, na forma que especifica.
	\end{subsubapend}
\end{subapend}



\begin{subapend}
	\subsection{Estadual}
	\begin{subsubapend}
		\item \subsubsection{Lei nº 9.338/1996}
		Institui nas escolas estaduais de 1º e 2º graus a “Semana da Gincana de Coleta de Lixo Reciclável”.
		\subsubsection{Lei nº 9.532/1997}
		Sobre a semana de coleta seletiva e reciclagem de lixo.
		\subsubsection{Lei nº 12.047/2005}
		Institui Programa Estadual de Tratamento e Reciclagem de Óleos e Gorduras de Origem Vegetal ou Animal e Uso Culinário.
		\subsubsection{Lei 13.576/2009}
		Sobre a destinação, reciclagem e gerenciamento do lixo tecnológico.
		\subsubsection{Lei nº 14.731/2012}
		Inclui evento no Calendário Oficial do Estado: "Dia dos Catadores de Lixo Reciclável", a ser comemorado, anualmente, em 20 de dezembro.
		\subsubsection{Lei nº 15.303/2014}
		Institui o Programa Estadual de Incentivo ao uso de matérias–primas e insumos derivados de materiais reciclados provenientes da indústria petroquímica.
		\subsubsection{Resolução SSE/SMA 49 /2007}
		Cria Grupo de Trabalho para propor um programa estadual de aproveitamento energético de resíduos sólidos urbanos e outros rejeitos da atividade econômica.
		\subsubsection{Resolução da SMA 79/2009}
		Estabelece diretrizes e condições para a operação e o licenciamento da atividade de tratamento térmico de resíduos sólidos em Usinas de Recuperação de Energia – URE.
		\subsubsection{Resolução da SMA 88/2013}
		Institui o Cadastro de Entidades de Catadores de Materiais Recicláveis, no âmbito do Estado de São Paulo.
		
	\end{subsubapend}
\end{subapend}


\begin{subapend}
	\subsection{Municipal}
	\begin{subsubapend}
		\item \subsubsection{Lei 571/83}
		Fica instituída a Feira de Artesanato no Município de Monteiro Lobato.
		\subsubsection{Lei 1.295/05}
		Locação de imóvel localizado na Rua Bernardino de Campos,nº271,destinado à Reciclagem de Lixo.
		\subsubsection{Lei 1.302/05}
		Sobre a reutilização de material reciclado,no âmbito da Administração Municipal,e dá outras providências.
		\subsubsection{Lei 1.322/06}
		Sobre a locação do imóvel destinado à Reciclagem de Lixo.
		\subsubsection{Lei 1.354/07}
		Sobre a locação de imóvel destinado à Reciclagem de Lixo.
		\subsubsection{Lei 1.392/08}
		Sobre a locação de imóvel destinado à Reciclagem de Lixo.
		\subsubsection{Lei 1.418/09}
		Sobre a locação de imóvel destinado ao Depósito de Reciclagem de Lixo.
		\subsubsection{Lei 1.541/13}
		Sobre a obrigatoriedade do uso de tapumes de folhas ou chapas de ferro ou alumínio em obras de construções ou reformas realizadas pelos Poderes Executivo e Legislativo e pelos órgãos estaduais.
		\subsubsection{Decreto 1.389/2013}
		Sobre a disponibilidade de bens “inservíveis” (por ex. 220 cadeiras estofadas) para geração de renda pelo Fundo Social pela Solidariedade do município.
	\end{subsubapend}
\end{subapend}



\section{Penalidades, Sanções Administrativas e Fiscalização}

\begin{subapend}
	\subsection{Federal}
	\begin{subsubapend}
		\item \subsubsection{Decreto 6.514/2008}
		Dispõe sobre as infrações e sanções administrativas ao meio ambiente, estabelece o processo administrativo federal para apuração destas infrações, e dá outras providências.
		\subsubsection{Lei 9605/1998}
		Dispõe sobre as sanções penais e administrativas derivadas de condutas e atividades lesivas ao meio ambiente, e dá outras providências.
	\end{subsubapend}
\end{subapend}



\begin{subapend}
	\subsection{Estadual}
	\begin{subsubapend}
		\item \subsubsection{Resolução da SMA 103/2012}
		Dispõe sobre a fiscalização do gerenciamento de resíduos de serviços de saúde.
		\subsubsection{Resolução da SMA 114/2010}
		Designa os integrantes do Grupo Técnico para elaboração e acompanhamento dos Planos Regionais de Gerenciamento de Resíduos Sólidos.
		\subsection{Municipal}
		\subsubsection{Decreto 753\\1998}
		Sobre definições de Infrações e atos lesivos à limpeza pública e suas penalidades (multas).
	\end{subsubapend}
\end{subapend}



\section{Saneamento Básico}

\begin{subapend}
	\subsection{Estadual}
	\begin{subsubapend}
		\item \subsubsection{Decreto 7.217/2010}
		Regulamenta a lei de saneamento básico.
		\subsubsection{Resolução Conama 05/1988}
		Dispõe sobre o licenciamento de obras de saneamento básico.
	\end{subsubapend}
\end{subapend}
\subsection{Estadual}
\begin{subapend}
	
	\begin{subsubapend}
		\item \subsubsection{Lei nº 7.750/1992}
		Sobre a política estadual de saneamento.
		\subsubsection{Lei nº 4.882/1985}
		Sobre o Saneamento Geral e despesas relacionadas.
		\subsubsection{Constituição Estadual}
		Estabelece políticas, ações e deveres de saneamento básico.
		\subsubsection{Lei nº 8.275/1993}
		Sobre a criação da Secretaria de Recursos Hídricos, Obras e Saneamento.
		\subsubsection{Lei nº 10.083/1998}
		Dispõe sobre o código sanitário do estado.
		\subsubsection{Lei nº 10.107/1968}
		Sobre o Fundo Estadual de Saneamento Básico.
		\subsubsection{Lei nº 11.364/2003}
		Altera a denominação da Secretaria de Estado de Recursos Hídricos, Saneamento e Obras, e autoriza o Poder Executivo a extinguir a Secretaria de Estado de Energia e dá providências correlatas.
		\subsubsection{Decreto nº 50.079/1968}
		Dispõe sobre a constituição do Centro Tecnológico de Saneamento Básico, prevista na Lei Estadual nº 10.107, de 8 de maio de 1968, e dá outras providências.
		\subsubsection{Decreto 52.455/2007}
		Aprova o regulamento da Agência Reguladora de Saneamento e Energia do Estado de São Paulo - ARSESP.
		\subsubsection{Decreto nº 52.895/2008}
		Autoriza a Secretaria de Saneamento e Energia a representar o Estado de São Paulo na celebração de convênios com Municípios paulistas, ou consórcio de Municípios, visando à elaboração de planos de saneamento básico e sua consolidação no Plano Estadual de Saneamento Básico.
		\subsubsection{Decreto 55.565/2010}
		Sobre a prestação de serviços públicos de saneamento básico relativos à limpeza urbana e ao manejo de resíduos sólidos urbanos no estado de São Paulo e dá providências correlatas.
	\end{subsubapend}
\end{subapend}

\begin{subapend}
	\subsection{Municipal}
	\begin{subsubapend}
		\item \subsubsection{Lei 351/69}
		Autoriza o Executivo Municipal a celebrar convênio com o Fundo Estadual de Saneamento Básico, destinado a receber auxílio do Estado para os serviços de tratamento de água do Município.
		\subsubsection{Lei 514/77}
		Autoriza o Poder Executivo a outorgar à Companhia de Saneamento Básico - Sabesp, concessão para a execução e exploração dos serviços de abastecimento de água e de coleta e destino final de esgotos sanitários no Município.
		\subsubsection{Lei 572/83}
		Fica o Executivo Municipal autorizado a celebrar convênio com a Secretaria da Saúde do Estado de São Paulo, visando a melhoria da assistência médica e sanitária da população Lobatense.
		\subsubsection{Lei 864/91}
		Firma convênio com a Companhia de Saneamento Básico do Estado de São Paulo - SABESP, para obras de implantação do sistema de coleta, tratamento e disposição final de esgotos sanitários a ser executado no Município.
		\subsubsection{Lei 1.441/09}
		Celebra convênio com o Estado de São Paulo, através da Secretaria de Saneamento e Energia, objetivando a elaboração do Plano Municipal de Saneamento Básico, e sua consolidação no Plano Estadual de Saneamento Básico, em conformidade com as diretrizes gerais instituídas pela Lei Federal nº 11.445, de 05 de janeiro de 2007.
		\subsubsection{Plano Municipal Integrado de Saneamento Básico - PLASAN123}
		Descrição do dados atuais de limpeza urbana e manejo de resíduos sólidos; Projeção da geração de resíduos, ações objetivas para o sistema de limpeza urbana e manejo de resíduos sólidos; Ações objetivas para o sistema de limpeza urbana e manejo de resíduos sólidos; Planejamento do sistema de limpeza urbana e manejo de resíduos sólidos; Indicadores de resíduos sólidos e Plano de ações de contingência e emergência de serviços de limpeza pública e manejo de resíduos sólidos urbanos.
		\subsubsection{Plano de Desenvolvimento Turístico Municipal de Monteiro Lobato}
		Resumo do plano de saneamento básico.
	\end{subsubapend}
\end{subapend}

\section{Catadores}


\begin{subapend}
	\subsection{Estadual}
	\begin{subsubapend}
		\item \subsubsection{Decreto 7.405/2010}
		Institui o Programa Pró-Catador;
	\end{subsubapend}
\end{subapend}

\begin{subapend}
	\subsection{Estadual}
	\begin{subsubapend}
		\item \subsubsection{Lei nº 14.731/2012}
		Inclui evento no Calendário Oficial do Estado: "Dia dos Catadores de Lixo Reciclável", a ser comemorado, anualmente, em 20 de dezembro.
		\subsubsection{Resolução da SMA 88/2013}
		Institui o Cadastro de Entidades de Catadores de Materiais Recicláveis, no âmbito do Estado de São Paulo.
		\subsection{Municipal}
		\subsubsection{Lei 629/86}
		Autoriza a celebração de convênio entre a União Federal, através da Secretaria Especial de Ação Comunitária da Presidência da República e a Prefeitura Municipal de Monteiro Lobato, visando a implantação de projetos comunitários.
		\subsubsection{Lei 1.650/2017}
		Institui o Plano Diretor de Monteiro Lobato. Possui diretrizes para a participação de catadores de resíduos sólidos.
		\subsubsection{Decreto 225 \ 1979}
		Comissão Municipal de Promoção Social, promove a inclusão econômica e social e organização de comunidades.
	\end{subsubapend}
\end{subapend}

\section{Consórcios Públicos e Cooperativismo}

\begin{subapend}
	\subsection{Federal}
	\begin{subsubapend}
		\item \subsubsection{Lei 5764/1971}
		Define a Política Nacional de Cooperativismo, institui o regime jurídico das sociedades cooperativas, e dá outras providências.
		\subsubsection{Lei 11.107/2005}
		Sobre os consórcios públicos e soluções compartilhadas de gestão.
		\subsubsection{Decreto 6.007/2007}
		Regulamenta a lei de consórcios públicos - 11107/2005.
		\subsubsection{Decreto nº 6.017/2007}
		Regulamenta a Lei nº 11.107, de 6 de abril de 2005, que dispõe sobre normas gerais de contratação de consórcios públicos.
	\end{subsubapend}
\end{subapend}

\begin{subapend}
	\subsection{Municipal}
	\begin{subsubapend}
		\item \subsubsection{Lei 390/70}
		Autoriza o Prefeito a celebrar convênio com municípios da região para constituição do CODIVAP (Conselho de Desenvolvimento Integrado do Vale do Paraíba.
		\subsubsection{Lei 1.088/97}
		Autoriza e estabelece as condições para o Executivo Municipal promover a participação do Município na constituição, instalação e funcionamento do Consórcio Intermunicipal para reestruturação e coordenação da gestão das atividades de obras e serviços viários nas esferas dos municípios consorciados e seus respectivos territórios, com a realização de todos os atos referentes à viabilização e efetivação das concessões de obras e serviços, com consonância com a vontade dos consorciados e com projetos globais de caráter geral encaminhados pelo Estado e União.
		\subsubsection{Lei 1.171/01}
		Autoriza a Prefeitura de Monteiro Lobato, a participar do Consórcio Intermunicipal para conservação e manutenção de vias públicas municipais.
		\subsubsection{Lei 1.563/13}
		Ratifica o Protocolo de Intenções que celebram entre si os Municípios de Santo Antônio do Pinhal, São Bento do Sapucaí, Monteiro Lobato, Tremembé e Campos do Jordão, visando à re-adequação legal do Consórcio Intermunicipal Serra da Mantiqueira - CISMA e dá outras providências.
		\subsubsection{Decreto 863/2002}
		Sobre o grupo de voluntários de combate à dengue - alguns resíduos podem acumular água; esses voluntários poderiam cooperar na gerenciamento desses materiais.
	\end{subsubapend}
\end{subapend}

\section{Caracterização de Resíduos}

\begin{subapend}
	\subsection{Federal}
	\begin{subsubapend}
		\item \subsubsection{NBR 1.298/1993}
		Sobre líquidos livres - Verificação em amostra de resíduos - Método de ensaio.
		\subsubsection{NBR 8.911/1985}
		Sobre solventes - Determinação de material não volátil - Método de ensaio.
		\subsubsection{NBR 10.004/2004}
		Sobre a classificação de resíduos.
		\subsubsection{NBR 10.005/2004}
		Sobre ensaio de lixiviado de resíduos.
		\subsubsection{NBR 10.006/2004}
		Sobre ensaio de solubilizado de resíduo.
		\subsubsection{NBR 10.007/2004}
		Sobre amostragem de resíduos.
		\subsubsection{NBR 13.227/2017}
		Sobre agrotóxicos e afins - Determinação de resíduo não volátil
		\subsubsection{NBR 13.237/2017}
		Especifica um método de ensaio para determinação do resíduo por peneiramento úmido de produtos agrotóxicos e afins.
		\subsubsection{NBR 13.999/2017}
		Sobre resíduos de papel, cartão, pastas celulósicas e madeira - Determinação do resíduo (cinza) após a incineração a 525ºC.
		\subsubsection{NBR 15.051/2004}
		Sobre laboratórios clínicos - Gerenciamento de resíduos.
	\end{subsubapend}
\end{subapend}

\begin{subapend}
	\subsection{Estadual}
	\begin{subsubapend}
		\item \subsubsection{Norma Cetesb L5.510/1982}
		Lixiviação de resíduos industriais: Método de Ensaio.
		\subsubsection{SS/SMA/SJDC-SP 1/2004}
		Estabelece classificação, as diretrizes básicas e o regulamento técnico sobre Resíduos de Serviços de Saúde Animal - R.S.S.A
	\end{subsubapend}
\end{subapend}

\section{Tratamento e Armazenamento/Acondicionamento de Resíduos}

\begin{subapend}
	\subsection{Federal}
	\begin{subsubapend}
		\item \subsubsection{Resolução Conama 01/1986}
		Sobre EIA/RIMA para empreendimentos modificadores do meio ambiente, como os aterros sanitários, processamento e destino final de resíduos tóxicos ou perigosos.
		\subsubsection{Resolução Conama 04/1995}
		Sobre a proibição de atividades de natureza perigosa que sejam foco de atração de aves, tais como os vazadouros de lixo nas áreas de segurança aeroportuárias (ASA).
		\subsubsection{Resolução Conama 237/1997}
		Sobre licenciamento de áreas para tratamento de resíduos sólidos.
		\subsubsection{Resolução Conama 264/1999}
		Sobre procedimentos, critérios e aspectos técnicos específicos de licenciamento ambiental para o coprocessamento de resíduos em fornos rotativos de clínquer para a fabricação de cimento.
		\subsubsection{Resolução Conama 316/2002}
		Sobre tratamento térmico de resíduos - alterada pela resolução 386/2006.
		\subsubsection{Resolução Conama 368/2006}
		Sobre Resíduos de cemitério. Altera dispositivos da Resolução nº 335, de 03 de abril de 2003, que dispõe sobre o licenciamento ambiental de cemitérios. Alterada pela Resolução nº 402, de 17 de novembro de 2008.
		\subsubsection{Resolução Conama 404/2008}
		Sobre aterro sanitário de pequeno porte para RSU.
		\subsubsection{Resolução Conama 411/2009}
		Sobre procedimentos para inspeção de indústrias consumidoras ou transformadoras de produtos e subprodutos florestais madeireiros de origem nativa, bem como os respectivos padrões de nomenclatura e coeficientes de rendimento volumétricos, inclusive carvão vegetal e resíduos de serraria - complementa a 379/2006 e foi alterada pela 474/2016.
		\subsubsection{Resolução Conama 467/2015}
		Sobre critérios para a autorização de uso de produtos ou de agentes de processos físicos, químicos ou biológicos para o controle de organismos ou contaminantes em corpos hídricos superficiais e dá outras providências.
		\subsubsection{Resolução Conama 474/2016}
		Altera a Resolução no 411/2009, que dispõe sobre procedimentos para inspeção de indústrias consumidoras ou transformadoras de produtos e subprodutos florestais madeireiros de origem nativa, bem como os respectivos padrões de nomenclatura e coeficientes de rendimento volumétricos, inclusive carvão vegetal e resíduos de serraria, e dá outras providências. Altera os arts. 6º e 9º e os anexos II, III e VII da Resolução 411/2009.
		\subsubsection{Resolução Conama 481/2017}
		Estabelece critérios e procedimentos para garantir o controle e a qualidade ambiental do processo de compostagem de resíduos orgânicos, e dá outras providências.
		\subsubsection{NBR 8.418/1984}
		Sobre a apresentação de projetos de aterros de resíduos industriais perigosos - Procedimento.
		\subsubsection{NBR 8.419/1996}
		Sobre procedimentos para a apresentação de projetos de aterro sanitário de resíduos sólidos urbanos.
		\subsubsection{NBR 9.191/2000}
		Sobre sacos plásticos para acondicionamento de resíduos.
		\subsubsection{NBR 10.157/1987}
		Aterros de resíduos perigosos - Critérios para projeto, construção e operação – Procedimento.
		\subsubsection{NBR 11.174/1990}
		Fixa as condições exigíveis para obtenção das condições mínimas necessárias ao armazenamento de resíduos classes II - não inertes e III - inertes, de forma a proteger a saúde pública e o meio ambiente.
		\subsubsection{NBR 11.175/1990}
		Incineração de resíduos sólidos perigosos - Padrões de desempenho – Procedimento.
		\subsubsection{NBR 12.235/1992}
		Sobre armazenamento de resíduos químicos.
		\subsubsection{NBR 13.230/2008}
		Estabelece os símbolos para identificação das resinas termoplásticas utilizadas na fabricação de embalagens e acondicionamento plásticos, visando auxiliar na separação e posterior reciclagem dos materiais de acordo com a sua composição
		\subsubsection{NBR 13.334/2007}
		Sobre requisitos para contentor metálico de 0,80 $m^3$, 1,2 $m^3$ e 1,6 $m^3$ para coleta de resíduos sólidos por coletores-compactadores de carregamento traseiro.
		\subsubsection{NBR 13.591/1996}
		Sobre a compostagem – Terminologia.
		\subsubsection{NBR 13.896/1997}
		Aterros de resíduos não perigosos - Critérios para projeto, implantação e operação.
		\subsubsection{NBR 13.968/1997}
		Esta Norma estabelece os procedimentos para a adequada lavagem de embalagens rígidas vazias de agrotóxicos que contiverem formulações miscíveis ou dispersíveis em água, classificadas como embalagens não-perigosas, para fins de manuseio, transporte e armazenagem.
		\subsubsection{NBR 13.999/2017}
		Sobre resíduos de Papel, cartão, pastas celulósicas e madeira - Determinação do resíduo (cinza) após a incineração a 525ºC.
		\subsubsection{NBR 14.283/1999}
		Sobre resíduos em solos - Determinação da biodegradação pelo método respirométrico.
		\subsubsection{NBR 14.599/2015}
		Requisitos de segurança para coletores-compactadores de carregamento traseiro e lateral.
		\subsubsection{NBR 14.719/2001}
		Estabelece os procedimentos para a destinação final das embalagens rígidas, usadas, vazias, adequadamente lavadas de acordo com a NBR 13968, que contiveram formulações de agrotóxicos miscíveis ou dispersíveis em água.
		\subsubsection{NBR 14.879/2011}
		Estabelece os critérios de definição dos volumes geométricos das caixas de carga e dos compartimentos de carga dos coletores-compactadores de resíduos sólidos de carregamento traseiro.
		\subsubsection{NBR 14.935/2003}
		Estabelece os procedimentos para a correta e segura destinação final das embalagens de agrotóxicos vazias, não laváveis, não lavadas, mal lavadas, contaminadas ou não, rígidas ou flexíveis, que não se enquadrem na ABNT NBR 14719.
		\subsubsection{NBR 15.051/2004}
		Laboratórios clínicos - Gerenciamento de resíduos.
		\subsubsection{Resolução Conama 307/2002}
		Sobre gerenciamento de RCC, alterada pelas resoluções 348/2004, 431/2011, 448/2012, 469/2015.
		\subsubsection{NBR 15.112/2004}
		Sobre resíduos da construção civil e resíduos volumosos; ATTs; diretrizes para projeto, implantação e operação.
		\subsubsection{NBR 15.113/2004}
		Sobre RCC e inertes - aterros.
		\subsubsection{NBR 15.448/2008}
		Especifica os requisitos e os métodos de ensaio para determinar a compostabilidade de embalagens plásticas, visando a revalorização de resíduos pós-consumo, por meio de apontamento das características de biodegradação aeróbia seguida da desintegração e impacto no processo de compostagem.
		\subsubsection{NBR 15.849/2010}
		Sobre requisitos para área de aterro sanitário de pequeno porte. 
		\subsubsection{NR 11}
		Sobre o Transporte, Movimentação, Armazenagem e Manuseio de Materiais. Pode ser importante para acondicionamento de resíduos químicos.
		\subsubsection{NR 15}
		Sobre Atividades e Operações Insalubres. Pode ser aplicável em trabalhos de coleta e tratamento de resíduos sólidos.
		\subsubsection{NR 17}
		Sobre Ergonomia no ambiente de trabalho, o que pode ser aplicado aos serviços de gerenciamento dos resíduos sólidos.
	\end{subsubapend}
\end{subapend}

\begin{subapend}
	\subsection{Estadual}
	\begin{subsubapend}
		\item \subsubsection{Lei nº 4.435/1984}
		Veda a instalação de depósito de lixo, usinas de beneficiamento de resíduos sólidos e aterros sanitários em área que especifica.
		\subsubsection{Lei nº 5.597/1987}
		Sobre o zoneamento industrial no estado de SãoPaulo.
		\subsubsection{Lei nº 10.478/1999}
		Dispõe sobre a adoção de medidas de defesa sanitária vegetal no âmbito do Estado.
		\subsubsection{Lei nº 12.047/2005}
		Institui Programa Estadual de Tratamento e Reciclagem de Óleos e Gorduras de Origem Vegetal ou Animal e Uso Culinário.
		\subsubsection{Lei nº 14.186/2010}
		Dispõe sobre a coleta, o recolhimento e o destino final das embalagens plásticas de óleos lubrificantes, e dá outras providências correlatas.
		\subsubsection{Lei nº 14.470/2011}
		Dispõe sobre a separação dos resíduos recicláveis descartados pelos órgãos e entidades da administração pública estadual, na forma que especifica.
		\subsubsection{Lei nº 15.276/2014}
		Dispõe sobre a destinação de veículos em fim de vida útil e dá outras providências.
		\subsubsection{Lei nº 15.313/2014}
		Dispõe sobre a proibição do uso, armazenamento e reparo de instrumentos de medição como esfigmomanômetros e termômetros contendo mercúrio e dá outras providências.
		\subsubsection{Lei nº 15.413/2014}
		Dispõe sobre tratamento térmico por cremação de animais mortos provenientes de estabelecimentos de ensino e pesquisa e de assistência à saúde veterinária sediados no Estado de São Paulo.
		\subsubsection{Decreto nº 50.079/1968}
		Dispõe sobre a constituição do Centro Tecnológico de Saneamento Básico, prevista na Lei Estadual nº 10.107, de 8 de maio de 1968, e dá outras providências.
		\subsubsection{Decreto nº 59.113/2013}
		Estabelece novos padrões de qualidade do ar e dá providências correlatas. Obs: estabelece benefícios para sistemas de tratamento de resíduos sólidos que tenham um bom controle de emissões gasosas.
		\subsubsection{Decreto nº 60.150/2014}
		Regulamenta a Lei nº 15.276, de 2014, que dispõe sobre a destinação de veículos em fim de vida útil.
		\subsubsection{Decisão Cetesb nº. 26/2003}
		Homologa a Norma Técnica P4.263 - Procedimento para Utilização de Resíduos em Fornos de Produção de Clínquer (Processo E/341/2003).
		\subsubsection{Decisão Cetesb nº. 27/2008}
		Dispõe sobre a aprovação do Procedimento para Utilização de Resíduos Perigosos da Indústria Têxtil em Caldeiras, no Estado de São Paulo.
		\subsubsection{Decisão Cetesb nº. 120/2009}
		Dispõe sobre recomendações para o licenciamento de empresas produtoras de matérias primas para a produção de micronutrientes, empresas fabricantes de micronutrientes e de empresas produtoras de fertilizantes ou misturadoras que utilizam micronutrientes.
		\subsubsection{Decisão Cetesb nº. 135/2007}
		Dispõe sobre a homologação da Norma Técnica E15.010 - Sistema de Tratamento Térmico Sem Combustão de Resíduos dos Grupos A e E - Procedimento - junho/2007 - e dá outras providências.
		\subsubsection{Decisão Cetesb nº. 145/2010}
		Dispõe sobre a aprovação do Procedimento de gerenciamento de resíduos de aparas de couro e de pó de rebaixadeira oriundos do curtimento ao cromo.
		\subsubsection{Decisão Cetesb nº. 152/2007}
		Dispõe sobre procedimentos para gerenciamento de areia de fundição.
		\subsubsection{Decisão Cetesb nº. 273/2010}
		Dispõe sobre a Homologação da Norma Técnica de Efluentes e Lodos Fluidos de Indústrias Cítricas - Critérios e Procedimentos para aplicação no solo agrícola.
		\subsubsection{Decisão Cetesb nº. 388/2010}
		Aprova premissas e diretrizes para a aplicação de resíduos e efluentes em solo agrícola no Estado de São Paulo.
		\subsubsection{Norma Cetesb O1.012/1985}
		Sobre o projeto e operação de aterros industriais para resíduos perigosos: Procedimento.
		\subsubsection{Norma Cetesb L1.022 2007}
		Sobre a utilização de produtos biotecnológicos para tratamento de efluentes líquidos, resíduos sólidos e recuperação de locais contaminados: Procedimento.
		\subsubsection{Série de normas Cetesb P4}
		Sobre gerenciamento de resíduos.
		\subsubsection{Norma Cetesb P4.231/2006}
		Sobre a vinhaça - Critérios e Procedimentos para Aplicação no Solo Agrícola. Norma Cetesb P4.262 (2004) Dispõe sobre procedimentos para utilização de resíduos em fornos de produção clinquer (processo E/341/2003) – dezembro de 2003.
		\subsubsection{Norma Cetesb P4.241 (Sem Data)}
		Norma para apresentação de projetos de aterros sanitários de resíduos urbanos.
		\subsubsection{Norma Cetesb P4.262/2007}
		Gerenciamento de resíduos químicos provenientes de estabelecimentos de serviço de saúde - procedimento.
		\subsubsection{Norma Cetesb P4.263/2003}
		Dispõe sobre procedimentos para utilização de resíduos em fornos de produção clinquer (processo E/341/2003) – dezembro de 2003.
		\subsubsection{Norma Cetesb L10.101/1988} 
		Sobre os resíduos sólidos industriais – tratamento no solo: Procedimento.
		\subsubsection{Norma Cetesb E15.010/2011}
		Sobre sistemas de tratamento térmico sem combustão de resíduos de serviços de saúde contaminados biologicamente: procedimento.
		\subsubsection{Norma Cetesb E15.011/2007}
		Sobre o sistema de Incineração de Resíduos de Serviços de Saúde - Procedimento.
		\subsubsection{Portaria SMA-SP  CVS nº 21/2008}
		Normas para gerenciamento de RSS.
		\subsubsection{Resolução da SMA-SP 15/2017}
		Dispõe sobre o licenciamento ambiental de empreendimento ou atividades relativas aos resíduos sólidos.
		\subsubsection{Resolução da SMA-SP 22/2007}
		Estabelece que os resíduos citados pela Conama 358/2005 devem ter estabelecimentos de tratamento licenciados pela Cetesb.
		\subsubsection{Resolução da SMA 33/2005}
		Dispõe sobre procedimentos para o gerenciamento e licenciamento ambiental de sistemas de tratamento e disposição final de resíduos de serviços de saúde humana e animal no Estado de São Paulo. Revoga a 31/2003.
		\subsubsection{Resolução da SMA 36/2012}
		Estabelece os procedimentos operacionais, define calendário de fechamento e dispõe sobre o método de valoração dos passivos ambientais aplicados no cálculo do Índice de Avaliação Ambiental, e dá providências correlatas vinculadas ao exercício do ciclo 2012, do Programa Município Verde Azul. Inclui: Índice da Qualidade de Aterro de Resíduos – IQR.
		\subsubsection{Resolução da SMA 38/2011}
		Estabelece a relação de produtos geradores de resíduos de significativo impacto ambiental, para fins do disposto no artigo 19, do Decreto Estadual nº 54645/2009, que regulamenta a Lei Estadual nº 12300/2006, e dá providências correlatas. Obs: contém exigências para o comércio de produtos farmacêuticos, cosméticos e de limpeza doméstica.
		\subsubsection{Resolução da SMA 43/2013}
		Estabelece os procedimentos operacionais do Programa Município Verde Azul, e dispõe sobre o método de valoração dos passivos ambientais aplicados no cálculo do Índice de Avaliação Ambiental. Inclui: Índice da Qualidade de Aterro de Resíduos – IQR.
		\subsubsection{Resolução SSE/SMA 49/2007}
		Cria Grupo de Trabalho para propor um programa estadual de aproveitamento energético de resíduos sólidos urbanos e outros rejeitos da atividade econômica.
		\subsubsection{Resolução da SMA 75/2008}
		Dispõe sobre licenciamento das unidades de armazenamento, transferência, triagem, reciclagem, tratamento e disposição final de resíduos sólidos de Classes IIA e IIB, classificados segundo a Associação Brasileira de Normas Técnicas – ABNT NBR 10.004, e dá outras providências.
		\subsubsection{Resolução da SMA 79/2009}
		Estabelece diretrizes e condições para a operação e o licenciamento da atividade de tratamento térmico de resíduos sólidos em Usinas de Recuperação de Energia – URE.
		\subsubsection{Resolução da SMA 81/2014}
		Estabelece diretrizes para implementação do Módulo Construção Civil do Sistema Estadual de Gerenciamento Online de Resíduos Sólidos – SIGOR, e dá providências correlatas.
		\subsubsection{Resolução da SMA 102/2012}
		Dispõe sobre dispensa de licenciamento ambiental para as atividades de compostagem e vermicompostagem em instalações de pequeno porte, sob condições determinadas.
		\subsubsection{Resolução da SMA 115/2013}
		Trata do estabelecimento de programas de responsabilidade pós-consumo para os medicamentos domiciliares, vencidos ou em desuso.
	\end{subsubapend}
\end{subapend}

\begin{subapend}
	\subsection{Municipal}
	\begin{subsubapend}
		\item \subsubsection{Lei Orgânica (Resolução 1/2007)}
		Cap. XVIII - o-) “ao uso e armazenamento dos agrotóxicos, seus componentes e afins, bem como, a coleta e ao controle diferenciado do lixo produzido por estes produtos”;
		\subsubsection{Lei Orgânica do Município de Monteiro Lobato. Promulgada em 1990 e atualizada em 2007.}
		Art. 98
		Estabelece procedimentos para a implantação de Planos de obras e serviços municipais.
		\subsubsection{Lei 865/91}
		Dispõe sobre doação de materiais de construção a famílias de baixa renda.
		\subsubsection{Lei 1.104/98}
		Institui o Programa Municipal de conservação de estradas rurais "Melhor Caminho".
		\subsubsection{Lei 1.417/09}
		Sobre a locação de imóvel destinado à marcenaria e depósito do Setor de Serviços Urbanos.
		\subsubsection{Decreto 99/1974}
		O município não cobrará taxas por serviços de limpeza em áreas urbanas durante período especificado. Importante para saber que, historicamente, o município não possui tradição de onerar o munícipe pelos serviços prestados.
	\end{subsubapend}
\end{subapend}

\section{Transporte e Coleta de Resíduos}

\begin{subapend}
	\subsection{Federal}
	\begin{subsubapend}
		\item \subsubsection{Lei 7802/1989}
		Sobre a pesquisa, a experimentação, a produção, a embalagem e rotulagem, o transporte, o armazenamento, a comercialização, a propaganda comercial, a utilização, a importação, a exportação, o destino final dos resíduos e embalagens, o registro, a classificação, o controle, a inspeção e a fiscalização de agrotóxicos, seus componentes e afins, e dá outras providências.
		\subsubsection{NBR 13.221/2010}
		Transporte terrestre de resíduos.
		\subsubsection{NBR 14.652/2013}
		Sobre coleta e transporte de RSS.
		\subsubsection{NBR 13.968/1997}
		Esta Norma estabelece os procedimentos para a adequada lavagem de embalagens rígidas vazias de agrotóxicos que contiveram formulações miscíveis ou dispersíveis em água, classificadas como embalagens não-perigosas, para fins de manuseio, transporte e armazenagem.
		\subsubsection{ABNT NBR 15051/2004}
		Esta Norma estabelece as especificações para o gerenciamento dos resíduos gerados em laboratório clínico. O seu conteúdo abrange a geração, a segregação, o acondicionamento, o tratamento preliminar, o tratamento, o transporte e a apresentação à coleta pública dos resíduos gerados em laboratório clínico, bem como a orientação sobre os procedimentos a serem adotados pelo pessoal do laboratório.
		\subsubsection{NBR 15.833/2010}
		Esta Norma prescreve os procedimentos para o transporte, armazenamento e desmonte com reutilização, recuperação dos materiais recicláveis e destinação final de resíduos dos aparelhos de refrigeração.
		\subsubsection{NBR 13.332/2010}
		Define os termos relativos ao coletor-compactador de resíduos sólidos, acoplado ao chassi de um veículo rodoviário, e seus principais componentes.
		\subsubsection{NBR 13.334/2007}
		Sobre requisitos para contentor metálico de 0,80 $m^3$, 1,2 $m^3$ e 1,6 $m^3$ para coleta de resíduos sólidos por coletores-compactadores de carregamento traseiro.
		\subsubsection{NBR 13.463/1995}
		Sobre a coleta de resíduos sólidos.
		\subsubsection{NBR 14.599/2015}
		Sobre requisitos de segurança para coletores-compactadores de carregamento traseiro e lateral.
		\subsubsection{NBR 14.652/2013}
		Sobre a coleta e transporte de RSS.
		\subsubsection{NBR 14.879/2011}
		Estabelece os critérios de definição dos volumes geométricos das caixas de carga e dos compartimentos de carga dos coletores-compactadores de resíduos sólidos de carregamento traseiro.
		\subsubsection{NBR 15.911/2011}
		Sobre o contentor móvel de plástico. Parte 2: Contentor de duas rodas, com capacidade de 120 L, 240 L e 360 L, destinado à coleta de resíduos sólidos urbanos (RSU) e de saúde (RSS) por coletor compactador. 
		\subsubsection{NR 11}
		Sobre o Transporte, Movimentação, Armazenagem e Manuseio de Materiais. Pode ser importante para acondicionamento de resíduos químicos.
		\subsubsection{NR 15}
		Sobre Atividades e Operações Insalubres. Pode ser aplicável em trabalhos de coleta e tratamento de resíduos sólidos.
		\subsubsection{NR 21}
		Sobre o Trabalho a Céu Aberto. Importante para o trabalho de coleta de resíduos a céu aberto.
		\subsubsection{Resolução Conama 275/2001}
		Sobre a identificação e código de cores de resíduos de coleta seletiva.
	\end{subsubapend}
\end{subapend}

\begin{subapend}
	\subsection{Estadual}
	\begin{subsubapend}
		\item \subsubsection{Lei nº. 2.252/1979}
		Altera a redação de dispositivos da Lei nº 440, de 24 de setembro de 1974, que dispõe sobre o Imposto de Circulação de Mercadorias, e dá providências correlatas.
		\subsubsection{Lei 6374/1989}
		Sobre tributos e impostos sobre circulação de mercadorias e transportes intermunicipais. Alterada pela lei 9176/1995.
		\subsubsection{Lei nº 7.452/1991}
		Sobre penalidades aos bens de uso comum rodoviário.
		\subsubsection{Lei nº 9.338/1996}
		Institui nas escolas estaduais de 1º e 2º graus a “Semana da Gincana de Coleta de Lixo Reciclável”.
		\subsubsection{Lei nº 9.532/1997}
		Sobre a semana de coleta seletiva e reciclagem de lixo.
		\subsubsection{Lei nº 10.306/1999}
		Sobre lixeiras seletivas em escolas públicas estaduais.
		\subsubsection{Lei nº 10.503/2000}
		Dispõe sobre poluição nas rodovias estaduais e dá outras providências.
		\subsubsection{Lei nº 10.856/2001}
		Cria o Programa de Coleta Seletiva de Lixo nas escolas públicas do Estado de São Paulo e dá outras providências.
		\subsubsection{Lei 12.528/07}
		Obriga a implantação do processo de coleta seletiva de lixo em "shopping centers" e outros estabelecimentos que especifica, do estado de São Paulo.
		\subsubsection{Lei nº 14.186/2010}
		Dispõe sobre a coleta, o recolhimento e o destino final das embalagens plásticas de óleos lubrificantes, e dá outras providências correlatas.
	\end{subsubapend}
\end{subapend}

\begin{subapend}
	\subsection{Municipal}
	\begin{subsubapend}
		\item \subsubsection{Lei Orgânica (Resolução 1/2007). Promulgada em 1990}                               
		Art. 98: Estabelece procedimentos para a implantação de Planos de obras e serviços municipais.
	\end{subsubapend}
\end{subapend}


                                
\section{Educação Ambiental}

\begin{subapend}
	\subsection{Federal}
	\begin{subsubapend}
		\item \subsubsection{Lei 9.795/1999}
		Dispõe sobre a educação ambiental, institui a Política Nacional de Educação Ambiental.
		\subsubsection{Decreto 4.281/2002}
		Regulamenta a Lei no 9.795, de 27 de abril de 1999, que institui a Política Nacional de Educação Ambiental, e dá outras providências.
		\subsubsection{Portaria Ministerial: 169/ 2012}
		Institui, no âmbito da Política Nacional de Educação Ambiental, o Programa de Educação Ambiental e Agricultura Familiar- PEAAF.
		\subsubsection{Resolução Conama 2/2012}
		Estabelece as Diretrizes Curriculares Nacionais para a Educação Ambiental.
		\subsubsection{Resolução Conama 422/2010}
		Estabelece diretrizes para as campanhas, ações e projetos de Educação Ambiental, conforme Lei no 9.795/1999.
		\subsubsection{Instrução Normativa do IBAMA 2/2012}
		Estabelece as bases técnicas para programas de educação ambiental apresentados como medidas mitigadoras ou compensatórias, em cumprimento às condicionantes das licenças ambientais emitidas pelo IBAMA.
		\subsubsection{Política Nacional de Educação Ambiental 9.795/99}
		As atividades vinculadas à Política Nacional de Educação Ambiental devem ser desenvolvidas na educação em geral e na educação escolar.
		\subsubsection{Proposta de Diretrizes Curriculares Nacionais para a Educação Ambiental – MEC}
		Proposta para oficializar as Diretrizes Curriculares Nacionais para a Educação Ambiental, sugerindo também a inserção da dimensão ambiental nos diferentes cursos de Ensino Superior e que, no curso de pedagogia e nas diferentes licenciaturas da Educação Superior (formação inicial de professores), a Educação Ambiental seja atividade curricular, disciplina ou projetos interdisciplinares, capaz de acrescentar à tal formação não apenas os conteúdos desta temática e a relação dela com as diversas áreas do conhecimento, mas uma formação crítica que fortaleça a postura ética, política e o papel social dos docentes para a construção do projeto de cidadania.
		\subsubsection{Programas}
		\subsubsection{Programa Nacional de Educação Ambiental (PRONEA)}
		Elaborado para assegurar, no âmbito educativo, a interação e a integração equilibradas das múltiplas dimensões da sustentabilidade ambiental – ecológica, social, ética, cultural, econômica, espacial e política – ao desenvolvimento do país, buscando o envolvimento e a participação social na proteção, recuperação e melhoria das condições ambientais e de qualidade de vida.
		\subsubsection{Programa Municípios Educadores Sustentáveis (MES)}
		Estimula espaços coletivos dos municípios como espaços educadores, que formem cidadãos para a construção cotidiana da sustentabilidade e para a participação na gestão pública, Promove ações que propiciem a educação dos indivíduos para atuarem e se auto-educarem contribuindo para a educação de outros na construção de sociedades sustentáveis, Estimula e apoiar em cada município a organização das instituições locais e a realização de parcerias para a construção de projetos educativos que conduzam à sustentabilidade e cria indicadores regionais e sistemas de avaliação que permitam o monitoramento dos municípios e a obtenção do Certificado de participação e do Selo Município Educador Sustentável.
	\end{subsubapend}
\end{subapend}

\begin{subapend}
	\subsection{Estadual}
	\begin{subsubapend}
		\item \subsubsection{Lei nº 9.338/1996}
		Institui nas escolas estaduais de 1º e 2º graus a “Semana da Gincana de Coleta de Lixo Reciclável”.
		\subsubsection{Lei nº 9.532/1997}
		Sobre a semana de coleta seletiva e reciclagem de lixo.
		\subsubsection{Lei nº 10.306/1999}
		Sobre lixeiras seletivas em escolas públicas estaduais.
		\subsubsection{Lei nº 10.522/2000}
		Autoriza o Poder Executivo a instituir o Programa de Desenvolvimento de Atividades de Pesquisa Discente sobre temas incorporados ao Projeto Pedagógico das Unidades Escolares de Ensino Médio.
		\subsubsection{Lei nº 10.856/2001}
		Cria o Programa de Coleta Seletiva de Lixo nas escolas públicas do Estado de São Paulo e dá outras providências.
		\subsubsection{Política Estadual de Educação Ambiental - Lei 12.780/2007}
		Criada em conformidade com os princípios e objetivos da Política Nacional de Educação Ambiental (PNEA), o Programa Nacional de Educação Ambiental (ProNEA) e a Política Estadual do Meio Ambiente.
		\subsubsection{Resolução da SMA 115/2013}
		Trata do estabelecimento de programas de responsabilidade pós-consumo para os medicamentos domiciliares, vencidos ou em desuso.
	\end{subsubapend}
\end{subapend}

\begin{subapend}
	\subsection{Municipal}
	\begin{subsubapend}
		\item \subsubsection{Lei Orgânica (Resolução 1/2007)}
		Artigo 169. cap VI.“Promover a educação ambiental em todos os níveis de ensino e a conscientização pública para a preservação do meio ambiente.”
		\subsubsection{Projeto de lei}
		Institui a Política Municipal de Educação Ambiental de Monteiro Lobato.
		\subsubsection{Lei 486/75}
		Autoriza a celebração de convênio com a Secretaria de Estado dos Negócios da Educação, objetivando o entrosamento de recursos e esforços para o incentivo da educação.
		\subsubsection{Lei 571/83}
		Fica instituída a Feira de Artesanato no Município de Monteiro Lobato.
		\subsubsection{Lei 629/86}
		Autoriza a celebração de convênio entre a União Federal, através da Secretaria Especial de Ação Comunitária da Presidência da República e a Prefeitura Municipal de Monteiro Lobato, visando a implantação de projetos comunitários.
		\subsubsection{Lei 1.068/97}
		Dispõe sobre a criação de Conselho Municipal de Educação do Município de Monteiro Lobato.
		\subsubsection{Decreto 87/1973}
		Promove a educação e alfabetização.
		\subsubsection{Decreto 863/2002}
		Sobre o grupo de voluntários de combate à dengue - alguns resíduos podem acumular água; esses voluntários poderiam cooperar no gerenciamento desses materiais.
	\end{subsubapend}
\end{subapend}

\section{Conservação da Qualidade Ambiental}

\begin{subapend}
	\subsection{Federal}
	\begin{subsubapend}
		\item \subsubsection{Constituição Federal de 1988}
		Art. 23 - Estabelece que compete à União, Estado e municípios zelar pela Constituição, evitar danos a patrimônios culturais e proteger as paisagens naturais e sítios arqueológicos;
		Art. 30 - Estabelece que compete aos municípios legislar sobre assuntos de interesse local e suplementar a legislação federal e estadual no que couber;
		Capítulo VI - Do meio ambiente. Estabelece incumbências ao Poder Público e à coletividade, de modo que todos possam ter assegurado o direito ao meio ambiente ecologicamente equilibrado e à qualidade de vida.
		\subsubsection{Lei 6803/1980}
		Dispõe sobre as diretrizes básicas para o zoneamento industrial nas áreas críticas de poluição, e dá outras providências.
		\subsubsection{Lei 9433/1997}
		Institui a Política Nacional de Recursos Hídricos, cria o Sistema Nacional de Gerenciamento de Recursos Hídricos, regulamenta o inciso XIX do art. 21 da Constituição Federal, e altera o art. 1º da Lei nº 8.001, de 13 de março de 1990, que modificou a Lei nº 7.990, de 28 de dezembro de 1989. Obs.: Possui recomendações para a gestão de Resíduos sólidos.
		\subsubsection{Lei 9985/2000}
		Sobre unidades de conservação e assuntos relacionados a essas áreas.
		\subsubsection{10.257/2001}
		Regulamenta os arts. 182 e 183 da Constituição Federal, estabelece diretrizes gerais da política urbana e dá outras providências.
		\subsubsection{Lei 12.651/2012}
		Revoga o novo código florestal - lei 4771/1965 -  e dá outras providências.
		\subsubsection{Lei 12.725/2012}
		Dispõe sobre o controle da fauna nas imediações de aeródromos.
		\subsubsection{Resolução Conama 357/2005}
		Dispõe sobre a classificação dos corpos de água e diretrizes ambientais para o seu enquadramento, bem como estabelece as condições e padrões de lançamento de efluentes, e dá outras providências. Alterada pelas Resoluções nº 370, de 06 de abril de 2006, nº 397, de 03 de abril de 2008, nº 410, de 04 de maio de 2009, e nº 430, de 13 de maio de 2011.
		\subsubsection{Resolução Conama 417/2009}
		Sobre parâmetros básicos para definição de vegetação primária e dos estágios sucessionais secundários da vegetação de Restinga na Mata Atlântica e dá outras providências - Complementada pelas Resoluções nº 437, nº 438, nº 439, nº 440, nº 441, nº 442, nº 443, nº 444, nº 445, nº 446, nº 447 e nº 453, de 2012.
		\subsubsection{Resolução Conama 423/2010}
		Sobre parâmetros básicos para identificação e análise da vegetação primária e dos estágios sucessionais da vegetação secundária nos Campos de Altitude associados ou abrangidos pela Mata Atlântica.
		\subsubsection{Resolução 430/2011}
		Dispõe sobre resíduos que não podem ser lançados em corpos de água.  Altera a resolução 357/2005.
		\subsubsection{Resolução Conama 460/2013}
		Altera a Resolução 420/2009, que dispõe sobre critérios e valores orientadores de qualidade do solo quanto à presença de substâncias químicas e dá outras providências - Altera a Resolução  nº 420/2009 e altera o prazo do art. 8º, e acrescenta novo parágrafo.
		\subsubsection{Resolução 463/2014}
		Sobre o controle ambiental de produtos destinados à remediação - Revoga a Resolução nº 314/2002.
		\subsubsection{Resolução Conama 473/2015}
		Prorroga os prazos previstos no §2º do art. 1º e inciso III do art. 5º da Resolução nº 428/2010, que dispõe no âmbito do licenciamento ambiental sobre a autorização do órgão responsável pela administração da Unidade de Conservação (UC), de que trata o § 3º do artigo 36 da Lei nº 9.985/2000, bem como sobre a ciência do órgão responsável pela administração da UC no caso de licenciamento ambiental de empreendimentos não sujeitos a EIA-RIMA e dá outras providências. Altera o §2º do art. 1º e inciso III do art. 5º da Resolução nº 428/2010.
		\subsubsection{NBR 8.834/1996}
		Estabelece os procedimentos adequados ao gerenciamento dos resíduos sólidos e as alternativas que podem ser usadas em casos de emergência, com vistas a preservar a saúde pública e a qualidade do meio ambiente.
		\subsubsection{NBR 14.725/2014}
		Sobre os riscos à saúde e ao meio ambiente, que substâncias químicas podem apresentar.
		\subsubsection{Política Nacional de Meio Ambiente Lei 6.938/81}
		Tem por objetivo a preservação, melhoria e recuperação da qualidade ambiental propícia à vida, visando assegurar, no País, condições ao desenvolvimento socioeconômico, aos interesses da segurança nacional e à proteção da dignidade da vida humana.
	\end{subsubapend}
\end{subapend}

\begin{subapend}
	\subsection{Estadual}
	\begin{subsubapend}
		\item \subsubsection{Lei nº 997/1976}
		Dispõe sobre o controle da poluição do meio ambiente. Alterada pela Lei nº 9.477/1996.
		\subsubsection{Lei nº 6.134/1988}
		Sobre a preservação de depósitos naturais de água.
		\subsubsection{Lei nº 7.663/1991}
		Sobre o Sistema de Gestão de Recursos Hídricos.
		\subsubsection{Lei nº 9.146/1995}
		Sobre a criação de mecanismos de compensação financeira para municípios nos casos que especifica e dá providências correlatas.
		\subsubsection{Lei 9866/1997}
		Dispõe sobre diretrizes e normas para a proteção e recuperação das bacias hidrográficas dos mananciais de interesse regional do Estado de São Paulo e dá outras providências.
		\subsubsection{Lei nº 11.220/2002}
		Dispõe sobre a instituição do Polo Turístico das Cidades Religiosas e dá outras providências. Obs: Exige que essas cidades turísticas tenham sua qualidade ambiental e turística preservada da degradação por lançamento de resíduos.
		\subsubsection{Lei 11160/2002}
		Dispõe sobre a criação do Fundo Estadual de Prevenção e Controle da Poluição – FECOP.
		\subsubsection{Lei nº 11.165/2002}
		Institui o Código de Pesca e Aquicultura do Estado. Possui exigências sobre a conservação da qualidade da água e lançamento de efluentes e resíduos sólidos.
		\subsubsection{Lei nº 13.577/2009}
		Dispõe sobre diretrizes e procedimentos para a proteção da qualidade do solo e gerenciamento de áreas contaminadas, e dá outras providências correlatas.
		\subsubsection{Decreto nº 8.468/1976}
		Aprova o Regulamento da Lei nº 997, de 31 de maio de 1976, que dispõe sobre a prevenção e o controle da poluição do meio ambiente.
		\subsubsection{Decreto nº 45.643/2001}
		Dispõe sobre a obrigatoriedade da aquisição pela Administração Pública Estadual de lâmpadas de maior eficiência energética e menor teor de mercúrio, por tipo e potência, e dá providências correlatas.
		\subsubsection{Decreto nº 47.397/2002}
		Dá nova redação ao Título V e ao Anexo 5 e acrescenta os Anexos 9 e 10, ao Regulamento da Lei nº 997, de 31 de maio de 1976, aprovado pelo Decreto nº 8.468, de 8 de setembro de 1976, que dispõe sobre a prevenção e o controle da poluição do meio ambiente. Alterado pelos Decretos 52.469/2007 e 50.753/2006.
		\subsubsection{Decreto nº 47.400/2002}
		Regulamenta dispositivos da Lei Estadual nº 9.509, de 20 de março de 1997, referentes ao licenciamento ambiental, estabelece prazos de validade para cada modalidade de licenciamento ambiental e condições para sua renovação, estabelece prazo de análise dos requerimentos e licenciamento ambiental, institui procedimento obrigatório de notificação de suspensão ou encerramento de atividade, e o recolhimento de valor referente ao preço de análise.
		\subsubsection{Decreto nº 59.263/2013}
		Regulamenta a Lei nº 13.577, de 8 de julho de 2009, que dispõe sobre diretrizes e procedimentos para a proteção da qualidade do solo e gerenciamento de áreas contaminadas, e dá providências correlatas.
		\subsubsection{Decisão de Diretoria no 103/2007}
		Sobre o procedimento para gerenciamento de áreas contaminadas.
		\subsubsection{Norma Cetesb L1.022/2007}
		Sobre a utilização de produtos biotecnológicos para tratamento de efluentes líquidos, resíduos sólidos e recuperação de locais contaminados: Procedimento.
		\subsubsection{Resolução da SMA 36/2012}
		Estabelece os procedimentos operacionais, define calendário de fechamento e dispõe sobre o método de valoração dos passivos ambientais aplicados no cálculo do Índice de Avaliação Ambiental, e dá providências correlatas vinculadas ao exercício do ciclo 2012, do Programa Município VerdeAzul. Inclui: Índice da Qualidade de Aterro de Resíduos – IQR.
		\subsubsection{Resolução da SMA 38/2011}
		Estabelece a relação de produtos geradores de resíduos de significativo impacto ambiental, para fins do disposto no artigo 19, do Decreto Estadual nº 54645/2009, que regulamenta a Lei Estadual nº 12300/2006, e dá providências correlatas. Obs: contém exigências para o comércio de produtos farmacêuticos, cosméticos e de limpeza doméstica.
		\subsubsection{Resolução da SMA 43/2013}
		Estabelece os procedimentos operacionais do Programa Município VerdeAzul, e dispõe sobre o método de valoração dos passivos ambientais aplicados no cálculo do Índice de Avaliação Ambiental. Inclui: Índice da Qualidade de Aterro de Resíduos – IQR.
	\end{subsubapend}
\end{subapend}

\begin{subapend}
	\subsection{Municipal}
	\begin{subsubapend}
		\item \subsubsection{Lei Orgânica do Município de Monteiro Lobato.} 
		Promulgada em 1990 e atualizada em 2007
		\subsubsection{Art. 179}
		Dá atribuições ao CONDEMA, principalmente em relação à preservação de fauna e flora.
		\subsubsection{Art. 172}
		Estabelece áreas de constante proteção no município.
		\subsubsection{Lei 765/89}
		Cria o Conselho Municipal de Defesa do Meio Ambiente.
		\subsubsection{Lei 781/89}
		Dá nova redação ao artigo 4º da Lei n º 765/89, do dia 04 de Setembro de 1989, que cria o Conselho Municipal de Defesa do Meio Ambiente.
		\subsubsection{Lei 1.009/94}
		Dispõe sobre regulamentação de plantios e reflorestamentos no Município de Monteiro Lobato e dá outras providências.
		\subsubsection{Lei 1.277/04}
		Sobre contratos específicos de recursos financeiros de Prevenção e Controle da Poluição (FECOP).
		\subsubsection{Lei 1.442/09}
		Dispõe sobre Estudo e Relatório de Impacto Ambiental nos projetos de edificações.
		\subsubsection{Lei 1.479/10}
		Afirma convênios como Estado de São Paulo e a executar pagamentos para a implantação de projetos de pagamentos por serviços ambientais.
		\subsubsection{Decreto 508/1989}
		Institui o Conselho de Defesa do Meio Ambiente, assim como suas atribuições e atividades.
		\subsubsection{Decreto 518/1989}
		Sobre a Instituição do Conselho Municipal de Proteção ao Meio Ambiente.
		\subsubsection{Decreto 555/1990}
		Sobre o regimento interno do conselho municipal de defesa do meio ambiente “CONDEMA”.
		\subsubsection{Decreto 863/2002}
		Sobre o grupo de voluntários de combate à dengue - alguns resíduos podem acumular água; esses voluntários poderiam cooperar na gerenciamento desses materiais.
		\subsubsection{Decreto 1.397/2013}
		Abertura de orçamento vigente ao setor de proteção ambiental.
	\end{subsubapend}
\end{subapend}

\section{Segurança Pessoal}

\begin{subapend}
	\subsection{Federal}
	\begin{subsubapend}
		\item \subsubsection{NR 6}
		Sobre equipamentos de proteção individual - úteis sempre que houver algum contato humano com os resíduos sólidos.
		\subsubsection{NR 12}
		Sobre a Segurança no Trabalho em Máquinas e Equipamentos. Pode ser importante, caso sejam utilizados equipamentos de tratamento dos resíduos sólidos.
		\subsubsection{NR 16}
		Sobre Atividades e Operações Perigosas. Aplicável, caso sejam adotados procedimentos que apresentem perigo ao trabalhador no manejo dos resíduos sólidos.
		\subsubsection{NR 17}
		Sobre Ergonomia no ambiente de trabalho, o que pode ser aplicado aos serviços de gerenciamento dos resíduos sólidos.
		\subsubsection{ ABNT NBR 9735/2017}
		Esta Norma estabelece o conjunto mínimo de equipamentos para emergências no transporte terrestre de produtos perigosos.
		\subsubsection{	ABNT NBR 16248/2013}
		Esta Norma especifica os números de escala e os requisitos de transmitância para filtros de proteção contra radiação ultravioleta.
		\subsubsection{ ABNT NBR 16249/2013}
		Esta Norma especifica os números de escala e os requisitos de transmitância para filtros de proteção contra radiação infravermelha.
		\subsubsection{ABNT NBR 15051/2004}
		Esta Norma estabelece as especificações para o gerenciamento dos resíduos gerados em laboratório clínico. O seu conteúdo abrange a geração, a segregação, o acondicionamento, o tratamento preliminar, o tratamento, o transporte e a apresentação à coleta pública dos resíduos gerados em laboratório clínico, bem como a orientação sobre os procedimentos a serem adotados pelo pessoal do laboratório.
	\end{subsubapend}
\end{subapend}

\section{Financiamento, Crédito, Facilitação ou Convênios}
\begin{subapend}
	\subsection{Federal}
	\begin{subsubapend}
		\item \subsubsection{Decreto 7.619/2011}
		Regulamenta a concessão de crédito presumido do Imposto sobre Produtos Industrializados - IPI na aquisição de resíduos sólidos.
	\end{subsubapend}
\end{subapend}

 \begin{subapend}
 	\subsection{Estadual}
 	\begin{subsubapend}
 		\item \subsubsection{Lei nº 16.260/2016}
 		Autoriza a Fazenda do Estado a conceder a exploração de serviços ou o uso, total ou parcial, de áreas em propriedades estaduais que especifica e dá outras providências correlatas. Obs: há obrigações especificadas com relação ao gerenciamento de resíduos sólidos.
 		\subsubsection{Decreto nº 57.479/2011}
 		Institui o Programa Estadual Água é Vida para localidades de pequeno porte predominantemente ocupadas por população de baixa renda, mediante utilização de recursos financeiros estaduais não reembolsáveis, destinados a obras e serviços de infraestrutura, instalações operacionais e equipamentos e dá providências correlatas.
 		\subsubsection{Decreto nº 59.260/2013}
 		Institui o Programa Estadual de apoio financeiro a ações ambientais, denominado Crédito Ambiental Paulista, e dá providências correlatas.
 		\subsubsection{Decreto nº 60.298/2014}
 		Introduz alterações no RICMS. Obs: dá incentivos ao uso de resíduos sólidos urbanos na produção de energia, biogás e biometano. Alterado pelo Decreto nº 61.104/2015.
 	\end{subsubapend}
 \end{subapend}
 

 \begin{subapend}
 	 \subsection{Municipal}
 	\begin{subsubapend}
 		\item \subsubsection{Lei 351/69}
 		Autoriza o Executivo Municipal a celebrar convênio com o Fundo Estadual de Saneamento Básico, destinado a receber auxílio do Estado para os serviços de tratamento de água do Município.
 		\subsubsection{Lei 435/73}
 		Dispõe sobre a arrecadação da Taxa do Cemitério.
 		\subsubsection{Lei 486/75}
 		Autoriza a celebração de convênio com a Secretaria de Estado dos Negócios da Educação, objetivando o entrosamento de recursos e esforços para o incentivo da educação.
 		\subsubsection{Lei 572/83}
 		Fica o Executivo Municipal autorizado a celebrar convênio com a Secretaria da Saúde do Estado de São Paulo, visando a melhoria da assistência médica e sanitária da população Lobatense.
 		\subsubsection{Lei 629/86}
 		Autoriza a celebração de convênio entre a União Federal, através da Secretaria Especial de Ação Comunitária da Presidência da República e a Prefeitura Municipal de Monteiro Lobato, visando a implantação de projetos comunitários.
 		\subsubsection{Lei 1.078/97}
 		Autoriza o Executivo Municipal a celebrar convênio de colaboração técnica com a Universidade do Estado de São Paulo, por intermédio da Fundação de apoio USP - FUSP, objetivando o Desenvolvimento de Estudos e Pesquisas em quase todos os campos do conhecimento humano.
 		\subsubsection{Lei 1.277/04}
 		Contratos específicos de recursos financeiros de Prevenção e Controle da Poluição (FECOP).
 		\subsubsection{Lei 1.291/05}
 		Locação de imóvel destinado à Diretoria do Meio Ambiente.
 		\subsubsection{Lei 1.441/09}
 		Celebra convênio com o Estado de São Paulo,através da Secretaria de Saneamento e Energia, objetivando a elaboração do Plano Municipal de Saneamento Básico, e sua consolidação no Plano Estadual de Saneamento Básico, em conformidade com as diretrizes gerais instituídas pela Lei Federal nº 11.445, de 05 de janeiro de 2007.
 		\subsubsection{Lei 1.479/10}
 		Afirma convênios como Estado de São Paulo e a executar pagamentos para a implantação de projetos de pagamentos por serviços ambientais.
 		\subsubsection{Decreto 968/2006}
 		Define os valores de créditos suplementares para secretarias, fundo municipal de saúde, fundo municipal de assistência social, serviços municipais urbanos e serviços de estrada de rodagem.
 		\subsubsection{Decreto 1.397/2013}
 		Sobre a abertura de orçamento vigente ao setor de proteção ambiental.
 	\end{subsubapend}
 \end{subapend}

\section{Requisitos Gerais}

\begin{subapend}
	\subsection{Federal}
	\begin{subsubapend}
		\item \subsubsection{Lei 6766/1979}
		Dispõe sobre o Parcelamento do Solo Urbano e dá outras Providências.
		\subsubsection{Lei 8666/1993}
		Sobre processos de licitação de governos.
		\subsubsection{Lei 12.305/2010}
		Política Nacional de Resíduos Sólidos.
		\subsubsection{Decreto 7.390/2010}
		Regulamenta a Política Nacional sobre mudança do Clima - estabelece que é importante ter reaproveitamento dos resíduos, principalmente recuperação do metano.
		\subsubsection{Decreto 7.404/2010}
		Regulamenta a lei 12305/2010.
		\subsubsection{Resolução Conama 313/2002}
		Sobre o inventário de Resíduos Sólidos. Revoga a resolução 6/1988.
		\subsubsection{Resolução Conama 330/2003}
		Institui a Câmara Técnica de Saúde, Saneamento Ambiental.
		\subsubsection{Recomendação 12/2008}
		Recomenda a adoção de práticas sustentáveis no âmbito da Administração Pública.
		\subsubsection{NR 24}
		Sobre Condições Sanitárias e de Conforto nos Locais de Trabalho. Importante para que o trabalho com resíduos seja realizado em boas condições.
		\subsubsection{NR 26}
		Sobre a Sinalização de Segurança em locais de trabalho. Importante para projetos que exijam sinalização de segurança.
		\subsubsection{IN do IBAMA 13/2012}
		Lista Brasileira de Resíduos Sólidos
		\subsubsection{IN do IBAMA 06/2013}
		Cadastro Técnico Federal de Atividades Potencialmente Poluidoras e Utilizadoras de Recursos Ambientais - CTF/APP.
		\subsubsection{IN do IBAMA 18/2014}
		Complementa a IN do Ibama 06/2013, descrevendo atividades tabeladas.
		\subsubsection{IN do IBAMA 12/2013}
		Procedimentos para importação de resíduos sólidos.
		\subsubsection{IN do IBAMA 06/2014}
		Regulamenta o relatório anual de atividades potencialmente poluidoras e utilizadoras de recursos ambientais.
		\subsubsection{IN do IBAMA 34/2008}
		Regulamento técnico de inspeção
		\subsubsection{Plano Nacional de Resíduos Sólidos}
		Projeto preliminar. O Plano mantém estreita relação com os Planos Nacionais de Mudanças do Clima (PNMC), de Recursos Hídricos (PNRH), de Saneamento Básico (Plansab) e de Produção e Consumo Sustentável (PPCS).
		\subsubsection{Política Nacional sobre Mudança do Clima Lei 12.187/09}
		Quando executadas ações políticas, sob a responsabilidade dos entes políticos e dos órgãos da administração pública, serão observados os princípios da precaução, da prevenção, da participação cidadã, do desenvolvimento sustentável e das responsabilidades comuns, porém diferenciadas, este último no âmbito internacional, e, quanto às medidas a serem adotadas na sua execução.
		\subsubsection{Manuais}
		Termo de referência para elaboração de Planos Municipais de saneamento básico (Ministério da saúde e Fundação Nacional da Saúde, 2012).
		
		Sugestões para elaboração do Plano Municipal ou Intermunicipal de RS (Banco do Brasil, 2011).
		Mecanismo de Desenvolvimento limpo nos empreendimentos de manejo de RSU (Ministério das Cidades, 2006).
		
		Guia para implantação da PNRS nos municípios brasileiros de forma justa e inclusiva (Rede Nossa São Paulo e Rede Social Brasileira por Cidades Justas e Sustentáveis, 2013)
	\end{subsubapend}
\end{subapend}

\begin{subapend}
	\subsection{Estadual}
	\begin{subsubapend}
		\item \subsubsection{Constituição Estadual}
		Estabelece políticas, ações e deveres de saneamento básico.
		\subsubsection{Lei nº 118/1973}
		Sobre a formação da CETESB. Alterada pela lei 13.542/2009.
		\subsubsection{Lei nº 119/1973}
		Sobre a constituição da Sabesp.
		\subsubsection{Lei nº 4.882/1985}
		Sobre o Saneamento Geral e despesas relacionadas.
		\subsubsection{Lei nº 7.750/1992}
		Sobre a política estadual de saneamento.
		\subsubsection{Lei nº 8.275/1993}
		Sobre a criação da Secretaria de Recursos Hídricos, Obras e Saneamento.
		\subsubsection{Lei nº 8.794/1994}
		Sobre a criação da CEAGESP.
		\subsubsection{Lei nº 10.083/1998}
		Dispõe sobre o código sanitário do estado.
		\subsubsection{Lei nº 11.364/2003}
		Altera a denominação da Secretaria de Estado de Recursos Hídricos, Saneamento e Obras, e autoriza o Poder Executivo a extinguir a Secretaria de Estado de Energia e dá providências correlatas.
		\subsubsection{Lei 11.387/2003}
		Sobre a apresentação, pelo Poder Executivo, de um Plano Diretor de Resíduos Sólidos para o Estado de São Paulo e dá providências correlatas.
		\subsubsection{Lei nº 13.507/2009}
		Dispõe sobre o Conselho Estadual de Meio Ambiente – CONSEMA, e dá providências correlatas.
		\subsubsection{Decreto nº 20.903/1983}
		Cria o Conselho Estadual do Meio Ambiente - CONSEMA.
		\subsubsection{Decreto nº 47.400/2002}
		Regulamenta dispositivos da Lei Estadual nº 9.509, de 20 de março de 1997, referentes ao licenciamento ambiental, estabelece prazos de validade para cada modalidade de licenciamento ambiental e condições para sua renovação, estabelece prazo de análise dos requerimentos e licenciamento ambiental, institui procedimento obrigatório de notificação de suspensão ou encerramento de atividade, e o recolhimento de valor referente ao preço de análise.
		\subsubsection{Decreto 52.455/2007}
		Aprova o regulamento da Agência Reguladora de Saneamento e Energia do Estado de São Paulo - ARSESP.
		\subsubsection{Decreto 54.645/2009}
		Regulamenta a lei estadual 12300/2006 - PERS. Alterado pelo Decreto nº 62.229/2016.
		\subsubsection{Decreto nº 55.947/2010}
		Regulamenta a Lei nº 13.798, de 9 de novembro de 2009, que dispõe sobre a Política Estadual de Mudanças Climáticas.
		\subsubsection{Decreto 57.071/2011}
		Altera a redação do "caput" do artigo 27 do Decreto nº 54.645, de 2009, que regulamenta dispositivos da Lei nº 12.300/2006, que instituiu a Política Estadual de Resíduos Sólidos.
		\subsubsection{Decreto nº 58.107/2012}
		Institui a estratégia para o Desenvolvimento sustentável do Estado de São Paulo 2020, e dá providências correlatas. Obs: Possui metas e estratégias de uso de resíduos sólidos até 2020.
		\subsubsection{Decreto nº 60.520/2014}
		Institui o Sistema Estadual de Gerenciamento Online de Resíduos Sólidos - SIGOR e dá providências correlatas.
		\subsubsection{Plano de Resíduos Sólidos do estado de São Paulo}
		Consta ações para apoio à gestão municipal de resíduos sólidos e às atividades de reciclagem, coleta seletiva e melhoria na destinação final dos resíduos sólidos; e na educação ambiental para a gestão de resíduos sólidos. Política Estadual de Saneamento Lei 7750/1992
		\subsubsection{Política Estadual de Meio Ambiente Lei nº 9.509/1997}
		Dispõe sobre a Política Estadual do Meio Ambiente, seus fins e mecanismos de formulação e aplicação.
		\subsubsection{Política Estadual de Resíduos Sólidos (PERS) Lei 12.300/2006}
		Define princípios e diretrizes, objetivos, instrumentos para a gestão integrada e compartilhada de resíduos sólidos, com vistas à prevenção e ao controle da poluição, à proteção e à recuperação da qualidade do meio ambiente, e à promoção da saúde pública, assegurando o uso adequado dos recursos ambientais no Estado de São Paulo.
		\subsubsection{Política Estadual de Mudanças Climáticas Lei nº 13.798/2009}
		Tem por objetivo geral estabelecer o compromisso do Estado frente ao desafio das mudanças climáticas globais, dispor sobre as condições para as adaptações necessárias aos impactos derivados das mudanças climáticas, bem como contribuir para reduzir ou estabilizar a concentração dos gases de efeito estufa na atmosfera.
		\subsubsection{Portaria SMA CVS nº 4/2011}
		Sobre o Sistema Estadual de Vigilância Sanitária.
		\subsubsection{Resolução da SMA 15/2017}
		Dispõe sobre o licenciamento ambiental de empreendimento ou atividades relativas aos resíduos sólidos.
		\subsubsection{Resolução da SMA 24/2016}
		Institui a Coordenação e os Comitês de Apoio Executivo à Gestão de Resíduos Sólidos do Sistema Ambiental Paulista, no âmbito da Secretaria de Estado do Meio Ambiente, a fim de integrar as ações relacionadas à Política Estadual de Resíduos Sólidos, e dá outras providências
		\subsubsection{Resolução da SMA 28/2016}
		Altera dispositivo da Resolução SMA nº 24, de 19 de fevereiro de 2016, que institui a Coordenação e os Comitês de Apoio Executivo à Gestão de Resíduos Sólidos do Sistema Ambiental Paulista, no âmbito da Secretaria de Estado do Meio Ambiente, a fim de integrar as ações relacionadas à Política Estadual de Resíduos Sólidos, e dá outras providências.
		\subsubsection{Resolução da SMA 65/2016}
		Altera o parágrafo único do artigo 2º e o artigo 5º da Resolução SMA nº 24, de 19 de fevereiro de 2016, que instituiu a Coordenação e os Comitês de Apoio Executivo à Gestão de Resíduos Sólidos do Sistema Ambiental Paulista, no âmbito da Secretaria de Estado do Meio Ambiente, a fim de integrar as ações relacionadas à Política Estadual de Resíduos Sólidos.
		\subsubsection{Resolução da SMA 91/2014}
		Instala Grupos de Trabalho para dar suporte às ações da Comissão Estadual de Gestão de Resíduos Sólidos.
	\end{subsubapend}
\end{subapend}

\begin{subapend}
	\subsection{Municipal}
	\begin{subsubapend}
		\item \subsubsection{Lei Orgânica do Município de Monteiro Lobato. Promulgada em 1990 e atualizada em 2007}
		Estabelece os fundamentos do Município, a organização dos Poderes, a organização administrativa municipal e a ordem econômica.
		\subsubsection{Art. 179} 
		Dá atribuições ao CONDEMA, principalmente em relação à preservação de fauna e flora.
		\subsubsection{Art. 98}
		Estabelece procedimentos para a implantação de Planos de obras e serviços municipais.
		\subsubsection{Lei 629/86}
		Autoriza a celebração de convênio entre a União Federal, através da Secretaria Especial de Ação Comunitária da Presidência da República e a Prefeitura Municipal de Monteiro Lobato, visando a implantação de projetos comunitários.
		\subsubsection{Lei 716/88}
		Dispõe sobre nova área de delimitação da Zona Urbana do Município e dá outras providências.
		\subsubsection{Lei 1.145/00}
		Dispõe sobre a delimitação do perímetro urbano.
		\subsubsection{Lei 1.291/05}
		Locação de imóvel destinado à Diretoria do Meio Ambiente.
		\subsubsection{Lei 1.443/09}
		Dispõe sobre divulgação das leis vigentes no Município.
		\subsubsection{Lei 1.445/09}
		Dispõe sobre a criação da Secretaria Municipal de Meio Ambiente e Agricultura e dá outras providências.
		\subsubsection{Lei 1.446/09}
		Estabelece a Agenda Ambiental Municipal do Meio Ambiente, e dá outras providências.
		\subsubsection{Lei 1.454/09}
		Dispõe sobre criação do Conselho Municipal de Meio Ambiente e do Fundo Municipal de Meio Ambiente e dá outras providências.
		\subsubsection{Lei 1.496/11}
		Declara e delimita Zonas Especiais de Interesse Social no Município de Monteiro Lobato e dá outras providências.
		\subsubsection{Lei 1.650/2017}
		Institui o Plano Diretor de Monteiro Lobato.
		\subsubsection{Lei 7/16}
		Plano Diretor de Desenvolvimento Sustentável do Turismo.
		\subsubsection{Decreto 225/1979}
		Comissão Municipal de Promoção Social, promove a inclusão econômica e social e organização de comunidades.
		\subsubsection{Decreto 508/1989}
		Institui o Conselho de Defesa do Meio Ambiente, assim como suas atribuições e atividades.
		\subsubsection{Decreto 518/1989}
		Instituição do Conselho Municipal de Proteção ao Meio Ambiente.
		\subsubsection{Decreto 555/1990}
		Sobre o regimento interno do conselho municipal de defesa do meio ambiente “CONDEMA”.
		\subsubsection{Decreto 833/2001}
		Instituição do Fórum pró Desenvolvimento Sustentável e Agenda 21, local.
		\subsubsection{Plano Municipal Integrado de Saneamento Básico - PLASAN123}
		Contém a descrição de dados atuais de limpeza urbana e manejo de resíduos sólidos; projeção da geração de resíduos, ações objetivas para o sistema de limpeza urbana e manejo de resíduos sólidos; ações objetivas para o sistema de limpeza urbana e manejo de resíduos sólidos; planejamento do sistema de limpeza urbana e manejo de resíduos sólidos; Indicadores de resíduos sólidos e um plano de ações de contingência e emergência de serviços de limpeza pública e manejo de resíduos sólidos urbanos.
		\subsubsection{Plano de Desenvolvimento Turístico Municipal de Monteiro Lobato}
		Contém um resumo do plano de saneamento básico.
	\end{subsubapend}
\end{subapend}

















